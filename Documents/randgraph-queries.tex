\chapter{Queries in Random Graph Experiments} 
\label{app:randgraph}

\begin{verbatim}
create table node (n integer);
insert into  node values (1);
insert into  node values (2);
insert into  node values (3);
insert into  node values (4);
......
insert into  node values (n-1);
insert into  node values (n); /* n is the number of nodes in the graph */

/* Here we specify the probability that an edge is in the graph. */
create table inout (bit integer, p float);
insert into  inout values (1, 0.5); /* probability that edge is in the graph */
insert into  inout values (0, 0.5); /* probability that edge is missing */

create table total_order as
(
   select n1.n as u, n2.n as v
   from node n1, node n2
   where n1.n < n2.n
);

/* This table represents all subsets of the total order over
   node as possible worlds. We use the same probability -- from
   inout -- for each edge, but in principle we could just as
   well have a different (independent) probability for each edge.
*/
create table to_subset as
(
   repair key u,v
   in (select * from total_order, inout)
   weight by p
);

create table edge0 as (select u,v from to_subset where bit=1);

select conf() as triangle_prob
from   edge0 e1, edge0 e2, edge0 e3
where  e1.v = e2.u and e2.v = e3.v and e1.u = e3.u
and    e1.u < e2.u and e2.u < e3.v;

select aconf(0.05,0.05) as triangle_prob
from   edge0 e1, edge0 e2, edge0 e3
where  e1.v = e2.u and e2.v = e3.v and e1.u = e3.u
and    e1.u < e2.u and e2.u < e3.v;

\end{verbatim}
\newpage
