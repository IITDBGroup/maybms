





\chapter{First Steps}


\section{Installing MayBMS}


\subsubsection{Using the installers}

Installers for MayBMS are available for both Windows and Linux operating systems and can be downloaded at

\url{https://sourceforge.net/projects/maybms/}

After you have obtained a copy of the installer, start it and follow the instructions.

\subsubsection{Compiling from scratch}

If the prepackaged installers do not work for you you can download the MayBMS's source code and compile it.
A copy is available for download at

\url{https://sourceforge.net/projects/maybms/}

Alternatively, you can obtain the latest snapshot from the repository by issuing the following command:
\begin{verbatim}
cvs -z3 -d:pserver:anonymous@maybms.cvs.sourceforge.net:/cvsroot/maybms
   co -P maybms 
\end{verbatim}
(on a single line).

This creates a directory maybms/ with a subdirectory postgresql-8.3.3/
that contains the source code of the system.

To compile and install MayBMS, just follow the instructions for installing
PostgreSQL 8.3.3. The latter is documented at

\medskip

\url{http://www.postgresql.org/docs/8.3/interactive/installation.html}.

%\medskip
%
%\noindent
%See
%
%\medskip
%
%\url{http://www.postgresql.org/docs/8.3/interactive/install-short.html}
%
%\medskip
%
%\noindent
%for a short version of the installation instructions.

\section{Running MayBMS}

After you have installed MayBMS (in either of the described ways), you can
set up a database and start using it.
Creating and accessing databases is the same as in PostgreSQL 8.3.3. Follow the links

\medskip

\url{http://www.postgresql.org/docs/8.3/interactive/tutorial-createdb.html}

\medskip

\noindent
and 

\medskip

\url{http://www.postgresql.org/docs/8.3/interactive/tutorial-accessdb.html}.

See next section for short instructions on how to run MayBMS.

\section{Short Instructions}


Alternatively, you can follow the following set of instructions.\footnote{If
you do know how to compile and install Postgres, or have followed the
installation instructions above, you can ignore this.}


On most UNIX machines, Postgres
is by default installed in the directory /usr/local/pgsql/
and run under user ``postgres''.
MayBMS uses the same defaults.
If you prefer to install MayBMS
in your home directory and run it with your user privileges, you do not
need root privileges to install it. Proceed as follows:
Change the path ac\_default\_prefix in line 279 of the file
maybms/postgresql-8.3.3/configure to a path into your home directory
(e.g. /home/myname/pgsql/ if your home directory is /home/myname/).

To compile, install, and start the Postgres server, execute the
following statements:
\begin{verbatim}
cd maybms/postgresql-8.3.3/
./configure
make
make install
cd
pgsql/bin/initdb -D mydbname
pgsql/bin/pg_cql start -D mydbname
\end{verbatim}

Note: In these minimal instructions, we did not create a special database
using createdb (so the default, template1, has to be used), and error
messages are written to the console.

Now MayBMS is available for connections from applications.

For example, the Postgres command line interface psql
in which you can issue MayBMS queries
and data manipulation language statements is started with
\begin{verbatim}
psql template1
\end{verbatim}
Now you can enter the examples from, e.g., the following tutorial.
The psql program is terminated using the command ``$\backslash$q''.
The database server is stopped with
\begin{verbatim}
pgsql/bin/pg_ctl stop -D mydbname
\end{verbatim}


\subsubsection{Remark}


Since Postgres and MayBMS use the same process identifiers,
MayBMS and Postgres cannot run concurrently on the same machine.
If you start Postgres when MayBMS is already running
(or vice versa), there will be an error message stating that Postgres is
already running. Since MayBMS always identifies itself as Postgres,
standard Postgres applications and middleware can run on MayBMS.  


\chapter{Probabilistic Databases}


We first give an informal definition of probabilistic databases, followed
by a formal definition.
%You should be able to do fine by just reading the
%first if the latter is too mathematical.


\section{Informal Definition}


Given a relational database schema (i.e., the structural information usually
specified by SQL CREATE TABLE statements).
A probabilistic database is a finite set of {\em possible worlds}\/, where each
possible world has a weight greater than 0 but no greater than 1
such that the sum of the weights
of all the worlds is one.
Each possible world is a relational database over the given
schema. That is, the schema is common to all possible worlds.

Possible worlds are a means of expressing uncertainty.
\begin{itemize}
\item
In a frequentist interpretation, the probabilistic database represents the
possible outcomes of a random experiment, the outcomes of which are relational
databases (or can be conveniently represented as relational databases).
The probability weight of a possible world is (the limit of) the relative
frequency of that possible world occurring as outcome of the random experiment
over a large number of trials.

\item
In a Bayesian interpretation, one of the possible worlds is
``true'', but we do not know which one, and the probabilities represent
degrees of belief in the various possible worlds.
\end{itemize}

Note that these interpretations of probabilistic databases are completely
standard in probability theory (and formalized via the notion of {\em probability spaces}). The only aspect particular to probabilistic
databases is the fact that possible worlds are relational databases.


Note that the idea of a probabilistic database as a set of possible worlds
is only the conceptual model. The physical representation of the
set of possible worlds in the MayBMS system is quite different
(see Section~\ref{sect:representation}) and allows for the efficient and
space-saving (compressed) representation of very large sets of possible
worlds.


\section{Formal Definition}


The following is a standard definition from probability theory and
shall only be recalled to demonstrate the close connection of
probabilistic databases to classical concepts in mathematics.


\begin{definition}\em
A {\em finite probability space}\/ is a triple $(\Omega, {\cal F}, \Pr)$ where
\begin{itemize}
\item
$\Omega$ is a finite set called the {\em sample space}\/,

\item
${\cal F} = 2^\Omega$ is the set of
subsets of $\Omega$ (these subsets are called {\em events}; the one-element subsets $\{\omega\}$ are called atomic events), and

\item
$\Pr$ is a
{\em probability measure}\/, i.e., a function
that maps each element $\omega \in \Omega$ (i.e., each atomic event)
to a number between 0 and 1 such that
\[
\sum_{\omega \in \Omega} \Pr[\omega] = 1
\]
and that maps
each (nonatomic) event $E \in ({\cal F} \;\backslash\; \Omega)$ to
$\sum_{\omega \in E} Pr[\omega]$.
\punto
\end{itemize}
\end{definition}


Formally,
a {\em probabilistic database}\/ over a relational database schema $sch$
is a finite probability space $(\Omega, {\cal F} = 2^\Omega, \Pr)$ with an associated
function $I$ (for {\em instance}) that maps each
$\omega \in \Omega$ to a relational database
over schema $sch$.

We call the elements $\omega$ of $\Omega$ the {\em possible worlds}\/
of the probabilistic database.

We can identify events with Boolean queries $Q$ that are true on a subset of
$\Omega$. Of course, the probability of such an event is given by
\[
\Pr[Q] = \sum_{\omega \in \Omega \;:\; Q(I(\omega))=true} \Pr[\omega].
\]

One particular type of event is membership of a given tuple $\vec{t}$
in the result of a (nonboolean) query, i.e., an event
\[
\{\omega \in \Omega \;:\; \vec{t} \in Q(I(\omega)) \}.
\]
The probability of this event is called the
{\em tuple confidence}\/ for tuple $\vec{t}$. 


A {\em random variable} $X$ is a function from $\Omega$ to a set $D$
(the ``values'' of the random variable).
We can associate each expression $X=x$, where $x \in D$, with an event
\[
\{ \omega \in \Omega \mid X(\omega) = x \}.
\]
Again, this is the usual notion from probability theory.


\section{An Example}


Consider a finite probability space with
\[
\Omega = \{
\omega_{rain, wet},
\omega_{\neg rain, wet}, 
\omega_{rain, \neg wet},
\omega_{\neg rain, \neg wet}
\}
\]
and
$\Pr[\omega_{rain, wet}] = 0.35$,
$\Pr[\omega_{rain, \neg wet}] = 0.05$,
$\Pr[\omega_{\neg rain, wet}] = 0.1$, and
$\Pr[\omega_{\neg rain, \neg wet}] = 0.5$.

Let $Wet$ be the event $\{ \omega_{rain, wet}, \omega_{\neg rain, wet} \}$.
Then $\Pr[Wet] = 0.35+0.1 = 0.45$.
%
We define Boolean random variables Wet and Rain as follows:
\[
Wet = \{
\omega_{rain, wet} \mapsto true,
\omega_{\neg rain, wet} \mapsto true, 
\omega_{rain, \neg wet} \mapsto false,
\omega_{\neg rain, \neg wet} \mapsto false
\};
\]\[ 
Rain = \{
\omega_{rain, wet} \mapsto true,
\omega_{\neg rain, wet} \mapsto false, 
\omega_{rain, \neg wet} \mapsto true,
\omega_{\neg rain, \neg wet} \mapsto false
\}.
\]
Then, $\Pr[Wet=true]$ is again $0.45$.

The first example of the following tutorial chapter captures this
example in the framework of the MayBMS query and update language.


\chapter{Tutorial}


This tutorial introduces the main features of MayBMS in an informal
way. The full examples can be run using the psql command line interface.



\section{A Really Simple Example}


We start by creating a simple table using SQL commands.
The table encodes that we see rain and wet ground with probability 0.4,
no rain but wet ground with probability 0.1, and
no rain and dry ground with probability 0.5.
%
\begin{verbatim}
create table R (Dummy varchar, Weather varchar,
                Ground varchar, P float);
insert into R values ('dummy',    'rain', 'wet', 0.35);
insert into R values ('dummy',    'rain', 'dry', 0.05);
insert into R values ('dummy', 'no rain', 'wet', 0.1);
insert into R values ('dummy', 'no rain', 'dry', 0.5);

select * from R;
 dummy | weather | ground |  p
-------+---------+--------+------
 dummy | rain    | wet    | 0.35
 dummy | rain    | dry    | 0.05
 dummy | no rain | wet    | 0.1
 dummy | no rain | dry    | 0.5
(4 rows)
\end{verbatim}

Table R is a completely standard relational database table,
created using standard SQL statement. One of the columns, P, stores
probabilities, but to the system these are only numbers without any particular
meaning so far.

The following statement creates a probabilistic database table S:
%
\begin{verbatim}
create table S as
repair key Dummy in R weight by P;
\end{verbatim}

The repair-key statement is one of the extensions of the MayBMS query
language over standard SQL, and it associates a special meaning to the
values taken from the ``weight by'' column.

The statement creates a probability space with a sample space consisting 
of three possible databases -- each one consisting just of one tuple
from $R$ -- with an associated probability measure given by the P column.

There are at least two natural interpretations of this example,
one using random variables and one using a possible worlds semantics.
%
\begin{itemize}
\item
We can think of S as a table specifying the joint probability distribution
of two discrete random variables Weather (with values ``rain'' and ``no rain'')
and Ground (with values ``wet'' and ``dry'').

\item
Alternatively, there are three possible worlds. Each of these worlds
is a relation S with a single tuple from R. The probability of such a world is
the value given for the tuple in column P of R.
\end{itemize}

We can compute the probabilities Pr[Ground='wet'] and
Pr[Weather='rain' and Ground='wet'] as follows
using the MayBMS conf() aggregate (which stands for ``confidence'').
%
\begin{verbatim}
create table Wet as
select conf() as P from S where Ground = 'wet';

select * from Wet;
  p
------
 0.45
(1 row)

create table Rain_and_Wet as
select conf() as P from S
where Weather = 'rain' and Ground = 'wet';

select * from Rain_and_Wet;
  p
------
 0.35
(1 row)
\end{verbatim}

The conditional probability
Pr[Weather='rain' $|$ Ground='wet'] can be computed as the ratio
\[
\mbox{Pr[Weather='rain' and Ground='wet'] / Pr[Ground='wet'].}
\]
%
\begin{verbatim}
select R1.P/R2.P as Rain_if_Wet from Rain_and_Wet R1, Wet R2;
 rain_if_wet
-------------
 0.777777778
(1 row)
\end{verbatim}

Since conf() is an aggregate, we can compute the marginal probability
table for random variable Ground as
\begin{verbatim}
select Ground, conf() from S group by Ground;
 ground | conf
--------+------
 dry    |  0.55
 wet    |  0.45
(2 rows)
\end{verbatim}


\section{Example: Triangles in Random Graphs}
\label{sec:randgraph}

In this tutorial, we compute the probability that a triangle occurs
in a random graph
with $k$ named (and thus distinguishable) nodes.
That is, we ask for the probability that an undirected
graph, chosen uniformly at random among the graphs of $k$ nodes,
contains at least one triangle.
This is equivalent to computing the count $n$ of graphs
that contain a triangle among the $2^{k \cdot (k-1)/2}$ undirected graphs of
$k$ distinguished nodes.
Indeed, an undirected graph of $k$ nodes has at most
$k \cdot (k-1)/2$ edges, and we obtain all the graphs over the given
$k$ nodes by considering all subsets of this maximal set of edges.


We start by creating a unary ``node'' relation, say with five nodes.
We do this with the standard SQL ``create table'' and ``insert'' commands,
which behave as usual in a relational database system.

\begin{verbatim}
create table node (n integer);
insert into  node values (1);
insert into  node values (2);
insert into  node values (3);
insert into  node values (4);
insert into  node values (5);
\end{verbatim}

Next we create the total order over the nodes, i.e., a binary relation
with exactly one edge between any two nodes. This is again a standard
SQL ``create table'' statement where we compute the tuples to be inserted
with a standard SQL query over the ``node'' relation.
%
\begin{verbatim}
create table total_order as
(
   select n1.n as u, n2.n as v
   from node n1, node n2
   where n1.n < n2.n
);
\end{verbatim}

We create a table to represent that each edge is either in the
graph (bit=1) or missing (bit=0).
%
\begin{verbatim}
create table inout (bit integer);
insert into  inout values (1);
insert into  inout values (0);
\end{verbatim}

The following operation introduces uncertainty into the database and
creates a probabilistic database with $2^{5 \cdot 4/2} = 1024$ possible
worlds, one for each possible edge relation over the five nodes
(=subset of the total order).
We do this by a query operation ``repair key'' that for each edge
of the total order nondeterministically chooses whether the edge is in
the graph (bit=1) or not. (That is, since we do not indicate at what
probability either of the two alternatives for bit is to be chosen, the
system makes the decision {\em uniformly}\/ at random, choosing bit=1 with
probability 0.5.) The resulting probabilistic database
represents all the alternative edge relations as possible worlds.
%
\begin{verbatim}
create table to_subset as
(
   repair key u,v in (select * from total_order, inout)
);
\end{verbatim}

The ``repair key'' operation is the most difficult to understand and
at the same time the most interesting addition to SQL that MayBMS provides.
Conceptually, ``repair key'' takes a set of attributes $\vec{K}$
and a relation $R$ (in this case the relational
product of total\_order and inout)
as arguments and nondeterministically chooses a maximal repair
of key $\vec{K}$ in $R$, that is, it removes a minimal set of tuples from
$R$ such that $\vec{K}$ ceases to violate a key constraint on columns $u, v$.
In this case, there are exactly two tuples for each pair $(u,v)$, namely
$(u,v,1)$ and $(u,v,0)$, and repair key chooses exactly one of them.
The consequence is that, overall, the operation nondeterministically
chooses a subset of the set of all edges. It chooses from these subsets
uniformly. The ``repair key'' operation accepts an additional argument that
allows us to assign nonuniform probabilities to the possible choices, but
in this case we do want uniform probabilities.

We have now created a probabilistic database. Conceptually, queries
and updates are evaluated in all possible worlds in parallel. Viewed differently, there is only one to\_subset relation (but we do not know which one), and
we continue to run queries and updates on this uncertain relation.

To actually create the edge relation, we select those tuples that have
bit=1 and compute their symmetric closure (to really represent an undirected
graph).
%
\begin{verbatim}
create table edge0    as (select u,v from to_subset where bit=1);

create table edge     as (select *              from edge0);
insert into  edge        (select v as u, u as v from edge0);
\end{verbatim}

Now we can compute the probability that the chosen graph has a triangle
as
%
\begin{verbatim}
select conf() as triangle_prob
from   edge e1, edge e2, edge e3
where  e1.v = e2.u and e2.v = e3.u and e3.v=e1.u
and    e1.u <> e2.u and e1.u <> e3.u and e2.u <> e3.u;
\end{verbatim}
where the conf aggregate computes the probability (``confidence'') that
the query given by the from-where statement returns a nonempty result.
This results in
%
\begin{verbatim}
 triangle_prob
---------------
      0.623355
(1 row)
\end{verbatim}

This is the correct probability: out of the 1024 possible
graphs of five nodes, 636 have a triangle, and $636/1024 \approx .623355$.
Indeed, the query
%
\begin{verbatim}
select *
from   edge e1, edge e2, edge e3
where  e1.v = e2.u and e2.v = e3.u and e3.v=e1.u
and    e1.u <> e2.u and e1.u <> e3.u and e2.u <> e3.u;
\end{verbatim}
%
computes at least one tuple in exactly those possible worlds (=on those
graphs) that have a triangle.
The conf() aggregate applied to this query conceptually
computes the sum of the probability
weights of the worlds in which the query has a nonempty result.
(The actual implementation does not naively iterate over possible
worlds, because this would be very inefficient.)

A more efficient implementation of the same query starts from the
``edge0'' relation:
%
\begin{verbatim}
select conf() as triangle_prob
from   edge0 e1, edge0 e2, edge0 e3
where  e1.v = e2.u and e2.v = e3.v and e1.u = e3.u
and    e1.u < e2.u and e2.u < e3.v;
\end{verbatim}

Finally, an even more efficient implementation uses the
aconf$(\epsilon, \delta)$ aggregate
to compute an $(\epsilon, \delta)$-approximation of the probability,
i.e., the probability that the computed value $\hat{p}$ returned by aconf
deviates from the
correct probability $p$ by more than $\epsilon \cdot p$ is less than $\delta$.
%
\begin{verbatim}
select aconf(.05,.05) as triangle_prob
from   edge0 e1, edge0 e2, edge0 e3
where  e1.v = e2.u and e2.v = e3.v and e1.u = e3.u
and    e1.u < e2.u and e2.u < e3.v;
\end{verbatim}

This result may be somewhat off, but the probability that the error is greater
than 5\% is less than 5\%.


Note that in the example we have seen only two extensions of SQL, ``repair
key'' and ``[a]conf''. The good news is that this is essentially
all there is. SQL extended by just these two features allows for very powerful
queries, including the computation of conditional probability tables,
maximum likelihood estimates, maximum-a-posteriori, Bayesian learning,
and much more.


\section{Example: Skills Management}


The following example demonstrates that probabilistic databases can be useful
even if the input data is not uncertain and the desired result is a classical
relational table. We define a hypothetical query in the context of skills
management. Assume we are given a classical relational database with two
tables, one, CE, stores possible takeover targets -- companies that we might
decide to buy with the employees that work in these companies.
The second table, ES, stores each employee's skills.

Here is an example database. We can build this database in MayBMS with the
standard SQL ``create table'' and ``insert'' statements.

\[
\begin{tabular}{l@{~}|@{~}l@{~~}l}
CE & CID & EID \\
\hline
  & Google & Bob  \\
  & Google & Joe  \\
  & Yahoo  & Dan  \\
  & Yahoo  & Bill \\
  & Yahoo  & Fred \\
\end{tabular}
\hspace{2mm}
\begin{tabular}{l@{~}|@{~}l@{~~}l}
ES & EID & Skill \\
\hline
  & Bob  & Web \\
  & Joe  & Web \\
  & Dan  & Java \\
  & Dan  & Web \\
  & Bill & Search \\
  & Fred & Java \\
\end{tabular}
\]


Now suppose that we want to buy exactly one of those companies, and
we expect exactly one employee to leave as a result of the takeover.
Which skills can we gain for certain?

We express this query in two steps. First we randomly choose a
company to buy and an employee who leaves, and compute the remaining
employees in the chosen company. We obtain this uncertain table using the
following query:
%
\begin{verbatim}
create table RemainingEmployees as
select CE.cid, CE.eid
from   CE,
       (repair key dummy
        in (select 1 as dummy, * from CE)) Choice
where  CE.cid =  Choice.cid
and    CE.eid <> Choice.eid;
\end{verbatim}

Note that the probabilistic database thus created contains five possible
worlds (since there are five tuples in CE), with a uniform probability
distribution. Not all these worlds have the same number of tuples: If we chose
Google and Bob, the world contains one tuple, Google and Joe. If we choose
Yahoo and Dan, the world contains two tuples, (Yahoo, Bill) and (Yahoo, Fred).

Now we compute which skills we gain for certain:
\begin{verbatim}
create table SkillGained as
select Q1.cid, Q1.skill, p1, p2, p1/p2 as p
from (select   R.cid, ES.skill, conf() as p1
      from     RemainingEmployees R, ES
      where    R.eid = ES.eid
      group by R.cid, ES.skill) Q1,
     (select cid, conf() as p2
      from RemainingEmployees
      group by cid) Q2
where Q1.cid = Q2.cid;

select cid, skill from SkillGained where p=1;
\end{verbatim}

The result is the table
\begin{center}
\begin{tabular}{ll}
CID & Skill \\
\hline
Google & Web \\
Yahoo  & Java \\
\end{tabular}
\end{center}
indicating that if we buy Google, we gain the skill ``Web'' for certain, and
if we buy Yahoo, we gain the skill ``Java'' for certain.

It is worth looking at the auxiliary table SkillGained:
\begin{center}
\begin{tabular}{l|l@{~~}l@{~~}l@{~~}l@{~~}l}
SkillGained & CID   & Skill  & p1  & p2  &    p     \\
\hline
& Google & Web    & 2/5 & 2/5 &   1 \\
& Yahoo  & Java   & 3/5 & 3/5 &   1 \\
& Yahoo  & Web    & 2/5 & 3/5 & 2/3 \\
& Yahoo  & Search & 2/5 & 3/5 & 2/3 \\
\end{tabular}
\end{center}

This table consists of the tuples $(x,y, p1, p2, p)$ such that
\begin{itemize}
\item
$x$ is a company,

\item
$y$ is a skill,

\item
$p1$ is the probability that the chosen company is $x$ and the skill
$y$ is gained (e.g., for $x$=Yahoo and $y$=Web, this is true in two of the
five possible worlds),

\item
$p2$ is the probability that $x$ is the chosen company
(e.g., for $x$=Yahoo, this is true in three of the five possible worlds), and

\item
$p=p1/p2$ is the probability that skill $y$ is gained if company $x$ is bought
(e.g., for $x$=Yahoo and $y$=Web, the probability is 2/3: of the three possible
worlds in which Yahoo was bought, only two worlds guarantee that the skill
Web is gained).
\end{itemize}

Thus, indeed, if we select those tuples of SkillGained for which $p=1$, we
obtain the desired pairs of companies and skills -- those skills that we
obtain for certain if we buy a company.


\section{Data Cleaning}


The following example is in the domain of data cleaning. Consider a
census in which a number of individuals complete forms, that are subsequently
digitized using an OCR system that will in some cases indicate a number
of alternative readings, together with probabilities.
For simplicity, let us assume that the forms only ask for a social security
number (SSN).

For example, if two individuals complete their forms and the OCR system
recognizes the SSN of the first to be either 185 (with probability .4) or
785 and the SSN of the second to be either 185 (with probability .7) or
186, we store this information in a probabilistic database constructed as
follows: 
%
\begin{verbatim}
create table Census_SSN_0 (tid integer, ssn integer, p float);

insert into Census_SSN_0 values (1, 185, .4);
insert into Census_SSN_0 values (1, 785, .6);
insert into Census_SSN_0 values (2, 185, .7);
insert into Census_SSN_0 values (2, 186, .3);

create table Census_SSN as
   repair key tid in Census_SSN_0 weight by p;
\end{verbatim}

We can view the alternatives and their probability weights by the following
query:
\begin{verbatim}
select   tid, ssn, conf() as prior
from     Census_SSN
group by tid, ssn;

 tid | ssn | prior
-----+-----+-------
   1 | 185 |   0.4
   1 | 785 |   0.6
   2 | 185 |   0.7
   2 | 186 |   0.3
\end{verbatim}

We can determine the probability that at least one individual has any
particular SSN (assuming that the OCR system did not miss the correct
SSN as an alternative) using the following query:
\begin{verbatim}
select   ssn, conf() as ssn_prior
from     Census_SSN
group by ssn;

 ssn | ssn_prior
-----+-----------
 185 |      0.82
 186 |       0.3
 785 |       0.6
\end{verbatim}
Indeed, the probability that at least one individual has SSN 185 is
$1 - .6 \cdot .3 = .82$.

We now perform data cleaning using a single integrity constraint, namely
that no two individuals can have the same ssn.
Conceptually, we want to exclude worlds that violate the functional dependency
\[
ssn \rightarrow tid,
\]
i.e., the constraint that ssn must be a key for the relation.

We start by computing an auxiliary relation that
computes, in each possible worlds, the ssn values that violate the
integrity constraint.
\begin{verbatim}
create table FD_Violations as
select S1.ssn
from   Census_SSN S1, Census_SSN S2
where  S1.tid < S2.tid and S1.ssn = S2.ssn;
\end{verbatim}
Note that two tuples violate the constraint if they have the same ssn
but different tid. We express this in the above query using a slightly
changed condition: (S1.tid $<$ S2.tid and S1.ssn = S2.ssn) instead of
(S1.tid $<>$ S2.tid and S1.ssn = S2.ssn). However, both conditions
select the same set of distinct ssn values that violate the integrity
constraint.

This query computes the uncertain table that holds 185 in the world in
which both forms have ssn value 185. In all other worlds it is empty.

Next we compute an auxiliary relation which computes, for each
SSN that occurs in at least one world in which an FD is violated,
the sum of the weights of those worlds in
which the SSN occurs and an FD is violated.
\begin{verbatim}

create table FD_Violations_by_ssn as
(
   select S.ssn, conf() as p
   from FD_Violations V,
        Census_SSN S
   group by S.ssn
);
\end{verbatim}

Next we compute the conditional probability table
\begin{verbatim}
create table TidSSNPosterior as
select Q1.tid, Q1.ssn, p1, p2, p3,
       cast((p1-p2)/(1-p3) as real) as posterior
from
   (
      select tid, ssn, conf() as p1
      from   Census_SSN
      group by tid, ssn
   ) Q1,
   (
      (select ssn, p as p2 from FD_Violations_by_ssn)
      union
      (
         (select ssn, 0 as p2 from Census_SSN_0)
         except
         (select possible ssn, 0 as p2 from FD_Violations_by_ssn)
      )
   ) Q2,
   (
      select conf() as p3
      from   FD_Violations
   ) Q3
where Q1.ssn = Q2.ssn;

select * from TidSSNPosterior;

 tid | ssn | p1  |  p2  |  p3  | posterior
-----+-----+-----+------+------+-----------
   1 | 185 | 0.4 | 0.28 | 0.28 |  0.166667
   1 | 785 | 0.6 |    0 | 0.28 |  0.833333
   2 | 185 | 0.7 | 0.28 | 0.28 |  0.583333
   2 | 186 | 0.3 |    0 | 0.28 |  0.416667
\end{verbatim}

This table stores, for each pair of form tid and ssn, the posterior probability
that the individual who completed the form tid has the social security number
ssn given that no two individuals can have the same ssn.

We can compute, for each form, the maximum-a-posteriori ssn (the most likely
ssn given the evidence specified by the integrity constraint) as
\begin{verbatim}
select tid, argmax(ssn, posterior) as map
from   TidSSNPosterior
group by tid
order by tid;

 tid | map
-----+-----
   1 | 785
   2 | 185
\end{verbatim}

In a sense, these map values are the locally best values that we could decide
upon for each uncertain answer in our census database. Note, however, that,
if we always choose the map value, we may sometimes create a database that
again violates the integrity constraints used for data cleaning.
This would have been the case if we had indicated probability .9 for both
185 alternatives in the input database.




A further example that computes conditional probabilities and MAP values
in a different context can be
found in Chapter~\ref{sect:ql} (Example~\ref{ex:coins_sql}).




