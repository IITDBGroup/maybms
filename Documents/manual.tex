\documentclass[12pt]{book}
\usepackage[bookmarks]{hyperref}
\usepackage{latexsym}
\usepackage{amssymb}
\usepackage{amsmath}
\usepackage{epsfig}
\usepackage{pst-tree}
\usepackage{multirow}
\usepackage{array}

%\addtolength{\textwidth}{1in}
%\addtolength{\oddsidemargin}{-0.5in}
%\addtolength{\evensidemargin}{-0.5in}
%\addtolength{\textheight}{1.5in}
%%\addtolength{\topmargin}{-1in}
%\addtolength{\topmargin}{-.2in}


\newtheorem{theorem}{Theorem}[section]
\newtheorem{metatheorem}{Metatheorem}[section]
\newtheorem{example}[theorem]{Example}
\newtheorem{algorithm}[theorem]{Algorithm}
\newtheorem{definition}[theorem]{Definition}
\newtheorem{proposition}[theorem]{Proposition}
\newtheorem{property}[theorem]{Property}
\newtheorem{corollary}[theorem]{Corollary}
\newtheorem{lemma}[theorem]{Lemma}
\newtheorem{remark}[theorem]{Remark}
\newtheorem{conjecture}[theorem]{Conjecture}
\newtheorem{proviso}[theorem]{Proviso}
\newtheorem{todo}[theorem]{ToDo}


\newcommand{\cby}[1]{#1}
\newcommand{\bluebox}[1]{#1}
\newcommand{\tuple}[1]{\langle #1 \rangle}
\newcommand{\nop}[1]{}
\newcommand{\difc}[1]{$#1$}
\def\lBrack{\lbrack\!\lbrack}
\def\rBrack{\rbrack\!\rbrack}
\newcommand{\Bracks}[1]{\lBrack#1\rBrack}
\def\punto{$\hspace*{\fill}\Box$}
\def\ph{\hat{p}}
\def\Pr{\mbox{Pr}}
\def\expec{\mathbf{E}}
\def\conf{\mathrm{conf}}
\def\rk{\mbox{repair-key}}


\title{MayBMS: A Probabilistic Database System \\[3ex]
User Manual
\\[6ex]
{\small Copyright (c) 2005-2009 \\
The MayBMS Development Group
\\[6ex]
Christoph Koch$^*$, Dan Olteanu$^{**}$, Lyublena Antova$^{*}$, and
Jiewen Huang$^{*,**}$ \\[4ex]
$^*$ Department of Computer Science,
Cornell University, Ithaca, NY \\[1ex]
$^{**}$ Oxford University Computing Laboratory, Oxford, UK}}

\author{}
\date{}


\renewcommand{\baselinestretch}{1.1}

\begin{document}


\maketitle

\tableofcontents

 
\chapter{Introduction}


\section{What is MayBMS?}


The {\em MayBMS}\/ system (note: MayBMS is read as ``maybe-MS'', like DBMS)
is a complete
probabilistic database management system that leverages robust
relational database technology:
MayBMS is an extension of the Postgres server backend.
MayBMS is open source and the source code
is available under the BSD license at
%
\begin{center}
http://maybms.sourceforge.net
\end{center}


The MayBMS system has been under development since 2005.
While the development has been carried out in an academic environment,
care has been taken to build a robust, scalable system that can be
reliably used in real applications.
%
The academic homepage of the MayBMS project is at

\begin{center}
http://www.cs.cornell.edu/database/maybms/
\end{center}



MayBMS stands alone as a complete probabilistic database management system
that supports a powerful, compositional query language for which nevertheless worst-case efficiency and result quality guarantees can be made.
We are aware of several research prototype probabilistic database management systems that are built as front-end applications of Postgres, but of no other fully integrated and available system. The MayBMS backend is accessible through several APIs, with efficient internal operators for computing and managing probabilistic data.


In summary, MayBMS has the following features:
\begin{itemize}
\item
Full support of all features of PostgreSQL 8.3.3, including unrestricted
query functionality, query optimization, APIs, updates, concurrency control and
recovery, etc.

\item
Essentially no performance loss on PostgreSQL 8.3.3 functionality:
After parsing a query or DML statement,
a fast syntactic check is made to decide
whether the statement uses the extended functionality of MayBMS. If it does
not, the subsequently executed code is exactly that of PostgreSQL 8.3.3.

\item
Support for efficiently creating and updating probabilistic databases,
i.e., uncertain databases in which degrees of belief can be associated
with uncertain data.

\item
A powerful query and update language for processing uncertain data
that gracefully extends SQL with a small number of well-designed
language constructs.

\item
State-of-the-art efficient techniques
for exact and approximate probabilistic inference.
\end{itemize}



\section{Applications}


Database systems for uncertain  and probabilistic data promise to have
many applications.  Query processing on  uncertain data occurs  in the
contexts of data warehousing, data integration, and of processing data
extracted from the Web. Data  cleaning can be fruitfully approached as
a problem of reducing uncertainty  in data and requires the management
and processing  of large amounts  of uncertain data.  Decision support
and   diagnosis  systems   employ   hypothetical  (what-if)   queries.
Scientific databases, which  store outcomes of scientific experiments,
frequently contain  uncertain data such as  incomplete observations or
imprecise   measurements.   Sensor  and   RFID   data  is   inherently
uncertain.  Applications   in  the  contexts  of   fighting  crime  or
terrorism,  tracking  moving  objects,  surveillance,  and  plagiarism
detection essentially  rely on techniques for  processing and managing
large  uncertain   datasets.  Beyond  that,   many  further  potential
applications  of  probabilistic  databases  exist  and  will  manifest
themselves once such systems become available.

The MayBMS distribution comes with a number of examples that illustrate
its use in
these application domains. Some of these examples are described in the
tutorial chapter of this manual.

The experiments section at the end of
this manual reports on some performance experiments with MayBMS. Unfortunately,
at the time of writing this, no benchmark for probabilistic database
systems exists, so these experiments are necessarily somewhat ad-hoc.



\section{Acknowledgments}


%MayBMS is an extension of PostgreSQL.

Michaela Goetz, Thomas Jansen and Ali Baran Sari are alumni of the MayBMS team.
%
The MayBMS project was previously supported by
German Science Foundation (DFG) grant KO 3491/1-1 and by funding provided by
the Center for Bioinformatics (ZBI) at Saarland University, Saarbruecken,
Germany. It is currently supported by grant IIS-0812272 of the
US National Science Foundation.














\chapter{First Steps}


\section{Installing MayBMS}


\subsubsection{Using the installers}

Installers for MayBMS are available for both Windows and Linux operating systems and can be downloaded at

\url{https://sourceforge.net/projects/maybms/}

After you have obtained a copy of the installer, start it and follow the instructions.

\subsubsection{Compiling from scratch}

If the prepackaged installers do not work for you you can download the MayBMS's source code and compile it.
A copy is available for download at

\url{https://sourceforge.net/projects/maybms/}

Alternatively, you can obtain the latest snapshot from the repository by issuing the following command:
\begin{verbatim}
cvs -z3 -d:pserver:anonymous@maybms.cvs.sourceforge.net:/cvsroot/maybms
   co -P maybms 
\end{verbatim}
(on a single line).

This creates a directory maybms/ with a subdirectory postgresql-8.3.3/
that contains the source code of the system.

To compile and install MayBMS, just follow the instructions for installing
PostgreSQL 8.3.3. The latter is documented at

\medskip

\url{http://www.postgresql.org/docs/8.3/interactive/installation.html}.

%\medskip
%
%\noindent
%See
%
%\medskip
%
%\url{http://www.postgresql.org/docs/8.3/interactive/install-short.html}
%
%\medskip
%
%\noindent
%for a short version of the installation instructions.

\section{Running MayBMS}

After you have installed MayBMS (in either of the described ways), you can
set up a database and start using it.
Creating and accessing databases is the same as in PostgreSQL 8.3.3. Follow the links

\medskip

\url{http://www.postgresql.org/docs/8.3/interactive/tutorial-createdb.html}

\medskip

\noindent
and 

\medskip

\url{http://www.postgresql.org/docs/8.3/interactive/tutorial-accessdb.html}.

See next section for short instructions on how to run MayBMS.

\section{Short Instructions}


Alternatively, you can follow the following set of instructions.\footnote{If
you do know how to compile and install Postgres, or have followed the
installation instructions above, you can ignore this.}


On most UNIX machines, Postgres
is by default installed in the directory /usr/local/pgsql/
and run under user ``postgres''.
MayBMS uses the same defaults.
If you prefer to install MayBMS
in your home directory and run it with your user privileges, you do not
need root privileges to install it. Proceed as follows:
Change the path ac\_default\_prefix in line 279 of the file
maybms/postgresql-8.3.3/configure to a path into your home directory
(e.g. /home/myname/pgsql/ if your home directory is /home/myname/).

To compile, install, and start the Postgres server, execute the
following statements:
\begin{verbatim}
cd maybms/postgresql-8.3.3/
./configure
make
make install
cd
pgsql/bin/initdb -D mydbname
pgsql/bin/pg_cql start -D mydbname
\end{verbatim}

Note: In these minimal instructions, we did not create a special database
using createdb (so the default, template1, has to be used), and error
messages are written to the console.

Now MayBMS is available for connections from applications.

For example, the Postgres command line interface psql
in which you can issue MayBMS queries
and data manipulation language statements is started with
\begin{verbatim}
psql template1
\end{verbatim}
Now you can enter the examples from, e.g., the following tutorial.
The psql program is terminated using the command ``$\backslash$q''.
The database server is stopped with
\begin{verbatim}
pgsql/bin/pg_ctl stop -D mydbname
\end{verbatim}


\subsubsection{Remark}


Since Postgres and MayBMS use the same process identifiers,
MayBMS and Postgres cannot run concurrently on the same machine.
If you start Postgres when MayBMS is already running
(or vice versa), there will be an error message stating that Postgres is
already running. Since MayBMS always identifies itself as Postgres,
standard Postgres applications and middleware can run on MayBMS.  


\chapter{Probabilistic Databases}


We first give an informal definition of probabilistic databases, followed
by a formal definition.
%You should be able to do fine by just reading the
%first if the latter is too mathematical.


\section{Informal Definition}


Given a relational database schema (i.e., the structural information usually
specified by SQL CREATE TABLE statements).
A probabilistic database is a finite set of {\em possible worlds}\/, where each
possible world has a weight greater than 0 but no greater than 1
such that the sum of the weights
of all the worlds is one.
Each possible world is a relational database over the given
schema. That is, the schema is common to all possible worlds.

Possible worlds are a means of expressing uncertainty.
\begin{itemize}
\item
In a frequentist interpretation, the probabilistic database represents the
possible outcomes of a random experiment, the outcomes of which are relational
databases (or can be conveniently represented as relational databases).
The probability weight of a possible world is (the limit of) the relative
frequency of that possible world occurring as outcome of the random experiment
over a large number of trials.

\item
In a Bayesian interpretation, one of the possible worlds is
``true'', but we do not know which one, and the probabilities represent
degrees of belief in the various possible worlds.
\end{itemize}

Note that these interpretations of probabilistic databases are completely
standard in probability theory (and formalized via the notion of {\em probability spaces}). The only aspect particular to probabilistic
databases is the fact that possible worlds are relational databases.


Note that the idea of a probabilistic database as a set of possible worlds
is only the conceptual model. The physical representation of the
set of possible worlds in the MayBMS system is quite different
(see Section~\ref{sect:representation}) and allows for the efficient and
space-saving (compressed) representation of very large sets of possible
worlds.


\section{Formal Definition}


The following is a standard definition from probability theory and
shall only be recalled to demonstrate the close connection of
probabilistic databases to classical concepts in mathematics.


\begin{definition}\em
A {\em finite probability space}\/ is a triple $(\Omega, {\cal F}, \Pr)$ where
\begin{itemize}
\item
$\Omega$ is a finite set called the {\em sample space}\/,

\item
${\cal F} = 2^\Omega$ is the set of
subsets of $\Omega$ (these subsets are called {\em events}; the one-element subsets $\{\omega\}$ are called atomic events), and

\item
$\Pr$ is a
{\em probability measure}\/, i.e., a function
that maps each element $\omega \in \Omega$ (i.e., each atomic event)
to a number between 0 and 1 such that
\[
\sum_{\omega \in \Omega} \Pr[\omega] = 1
\]
and that maps
each (nonatomic) event $E \in ({\cal F} \;\backslash\; \Omega)$ to
$\sum_{\omega \in E} Pr[\omega]$.
\punto
\end{itemize}
\end{definition}


Formally,
a {\em probabilistic database}\/ over a relational database schema $sch$
is a finite probability space $(\Omega, {\cal F} = 2^\Omega, \Pr)$ with an associated
function $I$ (for {\em instance}) that maps each
$\omega \in \Omega$ to a relational database
over schema $sch$.

We call the elements $\omega$ of $\Omega$ the {\em possible worlds}\/
of the probabilistic database.

We can identify events with Boolean queries $Q$ that are true on a subset of
$\Omega$. Of course, the probability of such an event is given by
\[
\Pr[Q] = \sum_{\omega \in \Omega \;:\; Q(I(\omega))=true} \Pr[\omega].
\]

One particular type of event is membership of a given tuple $\vec{t}$
in the result of a (nonboolean) query, i.e., an event
\[
\{\omega \in \Omega \;:\; \vec{t} \in Q(I(\omega)) \}.
\]
The probability of this event is called the
{\em tuple confidence}\/ for tuple $\vec{t}$. 


A {\em random variable} $X$ is a function from $\Omega$ to a set $D$
(the ``values'' of the random variable).
We can associate each expression $X=x$, where $x \in D$, with an event
\[
\{ \omega \in \Omega \mid X(\omega) = x \}.
\]
Again, this is the usual notion from probability theory.


\section{An Example}


Consider a finite probability space with
\[
\Omega = \{
\omega_{rain, wet},
\omega_{\neg rain, wet}, 
\omega_{rain, \neg wet},
\omega_{\neg rain, \neg wet}
\}
\]
and
$\Pr[\omega_{rain, wet}] = 0.35$,
$\Pr[\omega_{rain, \neg wet}] = 0.05$,
$\Pr[\omega_{\neg rain, wet}] = 0.1$, and
$\Pr[\omega_{\neg rain, \neg wet}] = 0.5$.

Let $Wet$ be the event $\{ \omega_{rain, wet}, \omega_{\neg rain, wet} \}$.
Then $\Pr[Wet] = 0.35+0.1 = 0.45$.
%
We define Boolean random variables Wet and Rain as follows:
\[
Wet = \{
\omega_{rain, wet} \mapsto true,
\omega_{\neg rain, wet} \mapsto true, 
\omega_{rain, \neg wet} \mapsto false,
\omega_{\neg rain, \neg wet} \mapsto false
\};
\]\[ 
Rain = \{
\omega_{rain, wet} \mapsto true,
\omega_{\neg rain, wet} \mapsto false, 
\omega_{rain, \neg wet} \mapsto true,
\omega_{\neg rain, \neg wet} \mapsto false
\}.
\]
Then, $\Pr[Wet=true]$ is again $0.45$.

The first example of the following tutorial chapter captures this
example in the framework of the MayBMS query and update language.


\chapter{Tutorial}


This tutorial introduces the main features of MayBMS in an informal
way. The full examples can be run using the psql command line interface.



\section{A Really Simple Example}


We start by creating a simple table using SQL commands.
The table encodes that we see rain and wet ground with probability 0.4,
no rain but wet ground with probability 0.1, and
no rain and dry ground with probability 0.5.
%
\begin{verbatim}
create table R (Dummy varchar, Weather varchar,
                Ground varchar, P float);
insert into R values ('dummy',    'rain', 'wet', 0.35);
insert into R values ('dummy',    'rain', 'dry', 0.05);
insert into R values ('dummy', 'no rain', 'wet', 0.1);
insert into R values ('dummy', 'no rain', 'dry', 0.5);

select * from R;
 dummy | weather | ground |  p
-------+---------+--------+------
 dummy | rain    | wet    | 0.35
 dummy | rain    | dry    | 0.05
 dummy | no rain | wet    | 0.1
 dummy | no rain | dry    | 0.5
(4 rows)
\end{verbatim}

Table R is a completely standard relational database table,
created using standard SQL statement. One of the columns, P, stores
probabilities, but to the system these are only numbers without any particular
meaning so far.

The following statement creates a probabilistic database table S:
%
\begin{verbatim}
create table S as
repair key Dummy in R weight by P;
\end{verbatim}

The repair-key statement is one of the extensions of the MayBMS query
language over standard SQL, and it associates a special meaning to the
values taken from the ``weight by'' column.

The statement creates a probability space with a sample space consisting 
of three possible databases -- each one consisting just of one tuple
from $R$ -- with an associated probability measure given by the P column.

There are at least two natural interpretations of this example,
one using random variables and one using a possible worlds semantics.
%
\begin{itemize}
\item
We can think of S as a table specifying the joint probability distribution
of two discrete random variables Weather (with values ``rain'' and ``no rain'')
and Ground (with values ``wet'' and ``dry'').

\item
Alternatively, there are three possible worlds. Each of these worlds
is a relation S with a single tuple from R. The probability of such a world is
the value given for the tuple in column P of R.
\end{itemize}

We can compute the probabilities Pr[Ground='wet'] and
Pr[Weather='rain' and Ground='wet'] as follows
using the MayBMS conf() aggregate (which stands for ``confidence'').
%
\begin{verbatim}
create table Wet as
select conf() as P from S where Ground = 'wet';

select * from Wet;
  p
------
 0.45
(1 row)

create table Rain_and_Wet as
select conf() as P from S
where Weather = 'rain' and Ground = 'wet';

select * from Rain_and_Wet;
  p
------
 0.35
(1 row)
\end{verbatim}

The conditional probability
Pr[Weather='rain' $|$ Ground='wet'] can be computed as the ratio
\[
\mbox{Pr[Weather='rain' and Ground='wet'] / Pr[Ground='wet'].}
\]
%
\begin{verbatim}
select R1.P/R2.P as Rain_if_Wet from Rain_and_Wet R1, Wet R2;
 rain_if_wet
-------------
 0.777777778
(1 row)
\end{verbatim}

Since conf() is an aggregate, we can compute the marginal probability
table for random variable Ground as
\begin{verbatim}
select Ground, conf() from S group by Ground;
 ground | conf
--------+------
 dry    |  0.55
 wet    |  0.45
(2 rows)
\end{verbatim}


\section{Example: Triangles in Random Graphs}
\label{sec:randgraph}

In this tutorial, we compute the probability that a triangle occurs
in a random graph
with $k$ named (and thus distinguishable) nodes.
That is, we ask for the probability that an undirected
graph, chosen uniformly at random among the graphs of $k$ nodes,
contains at least one triangle.
This is equivalent to computing the count $n$ of graphs
that contain a triangle among the $2^{k \cdot (k-1)/2}$ undirected graphs of
$k$ distinguished nodes.
Indeed, an undirected graph of $k$ nodes has at most
$k \cdot (k-1)/2$ edges, and we obtain all the graphs over the given
$k$ nodes by considering all subsets of this maximal set of edges.


We start by creating a unary ``node'' relation, say with five nodes.
We do this with the standard SQL ``create table'' and ``insert'' commands,
which behave as usual in a relational database system.

\begin{verbatim}
create table node (n integer);
insert into  node values (1);
insert into  node values (2);
insert into  node values (3);
insert into  node values (4);
insert into  node values (5);
\end{verbatim}

Next we create the total order over the nodes, i.e., a binary relation
with exactly one edge between any two nodes. This is again a standard
SQL ``create table'' statement where we compute the tuples to be inserted
with a standard SQL query over the ``node'' relation.
%
\begin{verbatim}
create table total_order as
(
   select n1.n as u, n2.n as v
   from node n1, node n2
   where n1.n < n2.n
);
\end{verbatim}

We create a table to represent that each edge is either in the
graph (bit=1) or missing (bit=0).
%
\begin{verbatim}
create table inout (bit integer);
insert into  inout values (1);
insert into  inout values (0);
\end{verbatim}

The following operation introduces uncertainty into the database and
creates a probabilistic database with $2^{5 \cdot 4/2} = 1024$ possible
worlds, one for each possible edge relation over the five nodes
(=subset of the total order).
We do this by a query operation ``repair key'' that for each edge
of the total order nondeterministically chooses whether the edge is in
the graph (bit=1) or not. (That is, since we do not indicate at what
probability either of the two alternatives for bit is to be chosen, the
system makes the decision {\em uniformly}\/ at random, choosing bit=1 with
probability 0.5.) The resulting probabilistic database
represents all the alternative edge relations as possible worlds.
%
\begin{verbatim}
create table to_subset as
(
   repair key u,v in (select * from total_order, inout)
);
\end{verbatim}

The ``repair key'' operation is the most difficult to understand and
at the same time the most interesting addition to SQL that MayBMS provides.
Conceptually, ``repair key'' takes a set of attributes $\vec{K}$
and a relation $R$ (in this case the relational
product of total\_order and inout)
as arguments and nondeterministically chooses a maximal repair
of key $\vec{K}$ in $R$, that is, it removes a minimal set of tuples from
$R$ such that $\vec{K}$ ceases to violate a key constraint on columns $u, v$.
In this case, there are exactly two tuples for each pair $(u,v)$, namely
$(u,v,1)$ and $(u,v,0)$, and repair key chooses exactly one of them.
The consequence is that, overall, the operation nondeterministically
chooses a subset of the set of all edges. It chooses from these subsets
uniformly. The ``repair key'' operation accepts an additional argument that
allows us to assign nonuniform probabilities to the possible choices, but
in this case we do want uniform probabilities.

We have now created a probabilistic database. Conceptually, queries
and updates are evaluated in all possible worlds in parallel. Viewed differently, there is only one to\_subset relation (but we do not know which one), and
we continue to run queries and updates on this uncertain relation.

To actually create the edge relation, we select those tuples that have
bit=1 and compute their symmetric closure (to really represent an undirected
graph).
%
\begin{verbatim}
create table edge0    as (select u,v from to_subset where bit=1);

create table edge     as (select *              from edge0);
insert into  edge        (select v as u, u as v from edge0);
\end{verbatim}

Now we can compute the probability that the chosen graph has a triangle
as
%
\begin{verbatim}
select conf() as triangle_prob
from   edge e1, edge e2, edge e3
where  e1.v = e2.u and e2.v = e3.u and e3.v=e1.u
and    e1.u <> e2.u and e1.u <> e3.u and e2.u <> e3.u;
\end{verbatim}
where the conf aggregate computes the probability (``confidence'') that
the query given by the from-where statement returns a nonempty result.
This results in
%
\begin{verbatim}
 triangle_prob
---------------
      0.623355
(1 row)
\end{verbatim}

This is the correct probability: out of the 1024 possible
graphs of five nodes, 636 have a triangle, and $636/1024 \approx .623355$.
Indeed, the query
%
\begin{verbatim}
select *
from   edge e1, edge e2, edge e3
where  e1.v = e2.u and e2.v = e3.u and e3.v=e1.u
and    e1.u <> e2.u and e1.u <> e3.u and e2.u <> e3.u;
\end{verbatim}
%
computes at least one tuple in exactly those possible worlds (=on those
graphs) that have a triangle.
The conf() aggregate applied to this query conceptually
computes the sum of the probability
weights of the worlds in which the query has a nonempty result.
(The actual implementation does not naively iterate over possible
worlds, because this would be very inefficient.)

A more efficient implementation of the same query starts from the
``edge0'' relation:
%
\begin{verbatim}
select conf() as triangle_prob
from   edge0 e1, edge0 e2, edge0 e3
where  e1.v = e2.u and e2.v = e3.v and e1.u = e3.u
and    e1.u < e2.u and e2.u < e3.v;
\end{verbatim}

Finally, an even more efficient implementation uses the
aconf$(\epsilon, \delta)$ aggregate
to compute an $(\epsilon, \delta)$-approximation of the probability,
i.e., the probability that the computed value $\hat{p}$ returned by aconf
deviates from the
correct probability $p$ by more than $\epsilon \cdot p$ is less than $\delta$.
%
\begin{verbatim}
select aconf(.05,.05) as triangle_prob
from   edge0 e1, edge0 e2, edge0 e3
where  e1.v = e2.u and e2.v = e3.v and e1.u = e3.u
and    e1.u < e2.u and e2.u < e3.v;
\end{verbatim}

This result may be somewhat off, but the probability that the error is greater
than 5\% is less than 5\%.


Note that in the example we have seen only two extensions of SQL, ``repair
key'' and ``[a]conf''. The good news is that this is essentially
all there is. SQL extended by just these two features allows for very powerful
queries, including the computation of conditional probability tables,
maximum likelihood estimates, maximum-a-posteriori, Bayesian learning,
and much more.


\section{Example: Skills Management}


The following example demonstrates that probabilistic databases can be useful
even if the input data is not uncertain and the desired result is a classical
relational table. We define a hypothetical query in the context of skills
management. Assume we are given a classical relational database with two
tables, one, CE, stores possible takeover targets -- companies that we might
decide to buy with the employees that work in these companies.
The second table, ES, stores each employee's skills.

Here is an example database. We can build this database in MayBMS with the
standard SQL ``create table'' and ``insert'' statements.

\[
\begin{tabular}{l@{~}|@{~}l@{~~}l}
CE & CID & EID \\
\hline
  & Google & Bob  \\
  & Google & Joe  \\
  & Yahoo  & Dan  \\
  & Yahoo  & Bill \\
  & Yahoo  & Fred \\
\end{tabular}
\hspace{2mm}
\begin{tabular}{l@{~}|@{~}l@{~~}l}
ES & EID & Skill \\
\hline
  & Bob  & Web \\
  & Joe  & Web \\
  & Dan  & Java \\
  & Dan  & Web \\
  & Bill & Search \\
  & Fred & Java \\
\end{tabular}
\]


Now suppose that we want to buy exactly one of those companies, and
we expect exactly one employee to leave as a result of the takeover.
Which skills can we gain for certain?

We express this query in two steps. First we randomly choose a
company to buy and an employee who leaves, and compute the remaining
employees in the chosen company. We obtain this uncertain table using the
following query:
%
\begin{verbatim}
create table RemainingEmployees as
select CE.cid, CE.eid
from   CE,
       (repair key dummy
        in (select 1 as dummy, * from CE)) Choice
where  CE.cid =  Choice.cid
and    CE.eid <> Choice.eid;
\end{verbatim}

Note that the probabilistic database thus created contains five possible
worlds (since there are five tuples in CE), with a uniform probability
distribution. Not all these worlds have the same number of tuples: If we chose
Google and Bob, the world contains one tuple, Google and Joe. If we choose
Yahoo and Dan, the world contains two tuples, (Yahoo, Bill) and (Yahoo, Fred).

Now we compute which skills we gain for certain:
\begin{verbatim}
create table SkillGained as
select Q1.cid, Q1.skill, p1, p2, p1/p2 as p
from (select   R.cid, ES.skill, conf() as p1
      from     RemainingEmployees R, ES
      where    R.eid = ES.eid
      group by R.cid, ES.skill) Q1,
     (select cid, conf() as p2
      from RemainingEmployees
      group by cid) Q2
where Q1.cid = Q2.cid;

select cid, skill from SkillGained where p=1;
\end{verbatim}

The result is the table
\begin{center}
\begin{tabular}{ll}
CID & Skill \\
\hline
Google & Web \\
Yahoo  & Java \\
\end{tabular}
\end{center}
indicating that if we buy Google, we gain the skill ``Web'' for certain, and
if we buy Yahoo, we gain the skill ``Java'' for certain.

It is worth looking at the auxiliary table SkillGained:
\begin{center}
\begin{tabular}{l|l@{~~}l@{~~}l@{~~}l@{~~}l}
SkillGained & CID   & Skill  & p1  & p2  &    p     \\
\hline
& Google & Web    & 2/5 & 2/5 &   1 \\
& Yahoo  & Java   & 3/5 & 3/5 &   1 \\
& Yahoo  & Web    & 2/5 & 3/5 & 2/3 \\
& Yahoo  & Search & 2/5 & 3/5 & 2/3 \\
\end{tabular}
\end{center}

This table consists of the tuples $(x,y, p1, p2, p)$ such that
\begin{itemize}
\item
$x$ is a company,

\item
$y$ is a skill,

\item
$p1$ is the probability that the chosen company is $x$ and the skill
$y$ is gained (e.g., for $x$=Yahoo and $y$=Web, this is true in two of the
five possible worlds),

\item
$p2$ is the probability that $x$ is the chosen company
(e.g., for $x$=Yahoo, this is true in three of the five possible worlds), and

\item
$p=p1/p2$ is the probability that skill $y$ is gained if company $x$ is bought
(e.g., for $x$=Yahoo and $y$=Web, the probability is 2/3: of the three possible
worlds in which Yahoo was bought, only two worlds guarantee that the skill
Web is gained).
\end{itemize}

Thus, indeed, if we select those tuples of SkillGained for which $p=1$, we
obtain the desired pairs of companies and skills -- those skills that we
obtain for certain if we buy a company.


\section{Data Cleaning}


The following example is in the domain of data cleaning. Consider a
census in which a number of individuals complete forms, that are subsequently
digitized using an OCR system that will in some cases indicate a number
of alternative readings, together with probabilities.
For simplicity, let us assume that the forms only ask for a social security
number (SSN).

For example, if two individuals complete their forms and the OCR system
recognizes the SSN of the first to be either 185 (with probability .4) or
785 and the SSN of the second to be either 185 (with probability .7) or
186, we store this information in a probabilistic database constructed as
follows: 
%
\begin{verbatim}
create table Census_SSN_0 (tid integer, ssn integer, p float);

insert into Census_SSN_0 values (1, 185, .4);
insert into Census_SSN_0 values (1, 785, .6);
insert into Census_SSN_0 values (2, 185, .7);
insert into Census_SSN_0 values (2, 186, .3);

create table Census_SSN as
   repair key tid in Census_SSN_0 weight by p;
\end{verbatim}

We can view the alternatives and their probability weights by the following
query:
\begin{verbatim}
select   tid, ssn, conf() as prior
from     Census_SSN
group by tid, ssn;

 tid | ssn | prior
-----+-----+-------
   1 | 185 |   0.4
   1 | 785 |   0.6
   2 | 185 |   0.7
   2 | 186 |   0.3
\end{verbatim}

We can determine the probability that at least one individual has any
particular SSN (assuming that the OCR system did not miss the correct
SSN as an alternative) using the following query:
\begin{verbatim}
select   ssn, conf() as ssn_prior
from     Census_SSN
group by ssn;

 ssn | ssn_prior
-----+-----------
 185 |      0.82
 186 |       0.3
 785 |       0.6
\end{verbatim}
Indeed, the probability that at least one individual has SSN 185 is
$1 - .6 \cdot .3 = .82$.

We now perform data cleaning using a single integrity constraint, namely
that no two individuals can have the same ssn.
Conceptually, we want to exclude worlds that violate the functional dependency
\[
ssn \rightarrow tid,
\]
i.e., the constraint that ssn must be a key for the relation.

We start by computing an auxiliary relation that
computes, in each possible worlds, the ssn values that violate the
integrity constraint.
\begin{verbatim}
create table FD_Violations as
select S1.ssn
from   Census_SSN S1, Census_SSN S2
where  S1.tid < S2.tid and S1.ssn = S2.ssn;
\end{verbatim}
Note that two tuples violate the constraint if they have the same ssn
but different tid. We express this in the above query using a slightly
changed condition: (S1.tid $<$ S2.tid and S1.ssn = S2.ssn) instead of
(S1.tid $<>$ S2.tid and S1.ssn = S2.ssn). However, both conditions
select the same set of distinct ssn values that violate the integrity
constraint.

This query computes the uncertain table that holds 185 in the world in
which both forms have ssn value 185. In all other worlds it is empty.

Next we compute an auxiliary relation which computes, for each
SSN that occurs in at least one world in which an FD is violated,
the sum of the weights of those worlds in
which the SSN occurs and an FD is violated.
\begin{verbatim}

create table FD_Violations_by_ssn as
(
   select S.ssn, conf() as p
   from FD_Violations V,
        Census_SSN S
   group by S.ssn
);
\end{verbatim}

Next we compute the conditional probability table
\begin{verbatim}
create table TidSSNPosterior as
select Q1.tid, Q1.ssn, p1, p2, p3,
       cast((p1-p2)/(1-p3) as real) as posterior
from
   (
      select tid, ssn, conf() as p1
      from   Census_SSN
      group by tid, ssn
   ) Q1,
   (
      (select ssn, p as p2 from FD_Violations_by_ssn)
      union
      (
         (select ssn, 0 as p2 from Census_SSN_0)
         except
         (select possible ssn, 0 as p2 from FD_Violations_by_ssn)
      )
   ) Q2,
   (
      select conf() as p3
      from   FD_Violations
   ) Q3
where Q1.ssn = Q2.ssn;

select * from TidSSNPosterior;

 tid | ssn | p1  |  p2  |  p3  | posterior
-----+-----+-----+------+------+-----------
   1 | 185 | 0.4 | 0.28 | 0.28 |  0.166667
   1 | 785 | 0.6 |    0 | 0.28 |  0.833333
   2 | 185 | 0.7 | 0.28 | 0.28 |  0.583333
   2 | 186 | 0.3 |    0 | 0.28 |  0.416667
\end{verbatim}

This table stores, for each pair of form tid and ssn, the posterior probability
that the individual who completed the form tid has the social security number
ssn given that no two individuals can have the same ssn.

We can compute, for each form, the maximum-a-posteriori ssn (the most likely
ssn given the evidence specified by the integrity constraint) as
\begin{verbatim}
select tid, argmax(ssn, posterior) as map
from   TidSSNPosterior
group by tid
order by tid;

 tid | map
-----+-----
   1 | 785
   2 | 185
\end{verbatim}

In a sense, these map values are the locally best values that we could decide
upon for each uncertain answer in our census database. Note, however, that,
if we always choose the map value, we may sometimes create a database that
again violates the integrity constraints used for data cleaning.
This would have been the case if we had indicated probability .9 for both
185 alternatives in the input database.




A further example that computes conditional probabilities and MAP values
in a different context can be
found in Chapter~\ref{sect:ql} (Example~\ref{ex:coins_sql}).









\chapter{Formal Foundations}


This chapter describes the formal foundations of MayBMS, including the
principles used for representing and storing probabilistic data, the
design of the query language, and efficient algorithms for query processing.

It is safe for a reader who has gained sufficient intuitive understanding
of the workings of MayBMS from the tutorial to skip this chapter on
first reading and to directly proceed to the query language reference
chapter that follows.

%To do: Say what the reader should at least skim over (e.g., the abstract
%definition of probabilistic databases) and what she can skip in the first
%reading and which can be looked up for reference. We may ultimately want
%to move this chapter to the end of the manual or into the appendix in
%order not to discourage the readers too much.


\section{Probabilistic Databases: Notation}
\label{sect:probdb}
\index{Probabilistic database}


\def\ww{{\bf W}}


Given a schema with relation names $R_1, \dots, R_k$. We use $sch(R_l)$ to denote the attributes of relation schema $R_l$.
Formally,
a {\em probabilistic database}\/ is a {\em finite}\/ set of structures
\[
\ww = \{ \tuple{R_1^1, \dots, R_k^1, p^{[1]}}, \dots,
         \tuple{R_1^n, \dots, R_k^n, p^{[n]}} \}
\]
of relations $R_1^i, \dots, R_k^i$ and numbers $0 < p^{[i]} \le 1$ such that
\[
\sum_{1 \le i \le n} p^{[i]} = 1.
\]
%
We call an element $\tuple{R_1^i, \dots, R_k^i, p^{[i]}} \in \ww$
a {\em possible world}\/, and $p^{[i]}$ its probability.
We use superscripts for indexing possible worlds.
To avoid confusion with exponentiation,
we sometimes use bracketed superscripts $\cdot^{[i]}$.
%
We call a relation $R$ {\em complete}\/ or {\em certain}\/
if its instantiations are the same in all possible worlds of $\ww$, i.e., if $R^1 = \cdots = R^n$.

Tuple {\em confidence}\/ refers to the probability of the event $\vec{t} \in R$, where $R$ is one of the relation names of the schema, with
\[
\Pr[\vec{t} \in R] = \sum_{1 \le i \le n:\; \vec{t} \in R^i} p^{[i]}.
\]









\section{Query Language Desiderata}
\label{sect:desiderata}
\index{Query language desiderata}


At the time of writing this, there is no accepted standard query language for probabilistic databases. In fact, we do not even agree today what use cases and functionality such systems should support.
It seems to be proper to start the query language discussion with the definition of design
{\em desiderata}\/. The following are those used in the design of MayBMS.
%
\begin{enumerate}
\item
Efficient query evaluation.

\item
The right degree of expressive power. The language should be powerful enough to support important queries. On the other hand, it should not be too strong, because expressiveness generally comes at a price: high evaluation complexity and infeasibility of query optimization.

\item
Genericity. The semantics of a query language should be independent from details of how the data is represented. Queries should behave in the same way no matter how the probabilistic data is stored. This is a basic requirement that is even part of the traditional definition of what constitutes a query (cf.\ e.g.\ \cite{AHV95}), but it is nontrivial to achieve for probabilistic databases \cite{AKO07ISQL}.

\item
The ability to transform data.
Queries on probabilistic databases are often interpreted quite narrowly in the literature.
%
%Probabilistic inference is sometimes understood as the problem of
%evaluating Boolean queries, or as ranking tuples by probability.
%
It is the authors' view that queries in general should be compositional mappings between databases, in this case probabilistic databases. This is a property taken for granted in relational databases. It allows for the definition of clean database update languages.

\item
The ability to introduce additional uncertainty.
This may appear to be a controversial goal, since uncertainty is commonly considered undesirable, and probabilistic databases are there to deal with it by providing useful functionality {\em despite}\/ uncertainty.
However, it can be argued that an uncertainty-introduction operation is important for at least three reasons:
(1)  for compositionality, and to allow construction of an uncertain database from scratch (as part of the update language);
(2) to support what-if queries; and
(3) to extend the hypothesis space modeled by the probabilistic database. The latter is needed to accommodate the results of experiments or new evidence, and to define queries that map from prior to posterior probabilistic databases. This is a nontrivial issue, and will be discussed in more detail later.
\end{enumerate}


The next section introduces a query algebra and argues that it satisfies each of these desiderata.









\section{The Algebra}
\label{sect:pwsa}
\index{Probabilistic world-set algebra}


This section covers the core query algebra of MayBMS: {\em probabilistic world-set algebra}\/ (probabilistic WSA) \cite{AKO07ISQL, Koch2008, Koch2008-SO}.
Informally, probabilistic world-set algebra consists of the operations of relational algebra,
an operation for computing tuple confidence conf, and the repair-key operation for {\em introducing}\/ uncertainty.
%
The operations of relational algebra are evaluated individually, in ``parallel'',
in each possible world.
The operation conf$(R)$
computes, for each tuple that occurs in relation $R$ in at least one world, the
sum of the probabilities of the worlds in which the tuple occurs.
The result is a certain relation, or viewed differently, a relation that is the same in
all possible worlds.
Finally, repair-key$_{\vec{A}@P}(R)$, where $\vec{A}, P$ are attributes of $R$,
conceptually nondeterministically chooses a maximal repair of
key $\vec{A}$.
This operation turns a possible world $R^i$ into the set of worlds consisting of all possible
{\em maximal repairs}\/ of key $\vec{A}$. A repair of key $\vec{A}$ in relation $R^i$ is a subset of $R^i$ for which $\vec{A}$ is a key.
It uses the numerically-valued column $P$ for weighting the newly created alternative
repairs.



Formally, probabilistic world-set algebra consists of the following operations:
\begin{itemize}
\item
The operations of relational algebra (selection $\sigma$, projection $\pi$,
product $\times$, union $\cup$, difference $-$, and attribute renaming $\rho$),
which are applied in each possible world independently.
\index{relational algebra}

The semantics of operations $\Theta$ on probabilistic database $\ww$ is
%
\[
\Bracks{\Theta(R_l)}(\ww) \; :=
\{ \tuple{R_1,\dots,R_k, \Theta(R_l), p}
\mid \tuple{R_1,\dots,R_k, p} \in \ww \}
\]
for unary operations ($1 \le l \le k$). For binary operations, the semantics is
\[
\Bracks{\Theta(R_l, R_m)}(\ww) \; :=
\{ \tuple{R_1,\dots,R_k, \Theta(R_l, R_m), p}
\mid \tuple{R_1,\dots,R_k, p} \in \ww \}.
\]

Selection conditions are Boolean combinations of atomic
conditions (i.e., negation is permitted even in the positive fragment of the algebra).
Arithmetic expressions may occur
in atomic conditions and in the arguments of $\pi$ and $\rho$. For example,
$\rho_{A+B \rightarrow C}(R)$ in each world
adds up the $A$ and $B$ values of each tuple
of $R$ and keeps them in a new $C$ attribute.



\item
An operation for computing tuple confidence,
\[
\Bracks{\mbox{conf}(R_l)}(\ww) :=
\{ \tuple{R_1,\dots,R_k, S, p} \mid \tuple{R_1,\dots,R_k, p} \in \ww \}
\]
where, w.l.o.g., $P \not\in sch(R_l)$, and
\[
S = \{ \tuple{\vec{t}, P: \Pr[\vec{t} \in R_l]} \mid
   \vec{t} \in \bigcup_i R_l^i \},
\]
with schema $sch(S) = sch(R_l) \cup \{ P \}$.
%
%Note that $S$ is a single certain relation with a
%column $P$ for holding probability values, rather than a probabilistic database.
%
The result of $\mbox{conf}(R_l)$, the relation $S$, is the same in all possible worlds, i.e., it is a certain relation.

By our definition of probabilistic databases, each possible world has nonzero probability. As a
consequence, conf does not return tuples with probability 0.

For example, on probabilistic database
\begin{center}
\begin{tabular}{c@{~~~}c@{~~~}c}
\begin{tabular}{@{~}c@{~}|@{~}c@{~~}c@{~}}
\hline
$R^{1}$ & A & B \\
\hline
& a & b \\
& b & c \\
\end{tabular}
$p^{[1]} = .3$
&
\begin{tabular}{@{~}c@{~}|@{~}c@{~~}c@{~}}
\hline
$R^{2}$ & A & B \\
\hline
& a & b \\
& c & d \\
\end{tabular}
$p^{[2]} = .2$
&
\begin{tabular}{@{~}c@{~}|@{~}c@{~~}c@{~}}
\hline
$R^{3}$ & A & B \\
\hline
& a & c \\
& c & d \\
\end{tabular}
$p^{[3]} = .5$
\end{tabular}
\end{center}
%
conf($R$) computes, for each possible tuple, the sum of the weights of the
possible worlds in which it occurs, here
%
\begin{center}
\begin{tabular}{c|c@{~~}c@{~~}c}
\hline
conf$(R)$ & $A$ & $B$ & P \\
\hline
& a & b & .5 \\
& a & c & .5 \\
& b & c & .3 \\
& c & d & .7 \\
\end{tabular}
\end{center}



\item
An uncertainty-introducing operation,
{\em repair-key}\/, which can be thought of as sampling a maximum repair of
a key for a relation.
Repairing a key of a complete relation $R$ means to compute, as possible worlds,
all subset-maximal relations
obtainable from $R$ by removing tuples such that a key constraint is satisfied.
We will use this as a method for constructing probabilistic databases,
with probabilities derived from relative weights attached to the tuples of $R$.
\index{Key repair}

We say that relation $R'$ is a {\em maximal repair}\/ of a functional dependency (fd, cf.\ \cite{AHV95}) for relation $R$ if $R'$ is a maximal subset of $R$ which satisfies that functional dependency, i.e., a subset $R' \subseteq R$ that satisfies the fd
such that there is no relation $R''$ with $R' \subset R'' \subseteq R$ that satisfies the fd.


Let $\vec{A}, B \in sch(R_l)$.
For each possible world $\tuple{R_1, \dots, R_k, p} \in \ww$,
let column $B$ of $R$ contain only numerical values greater than 0
and
let $R_l$ satisfy the fd $(sch(R_l) - B) \rightarrow sch(R_l)$.
Then,
\begin{multline*}
\Bracks{\rk_{\vec{A}@B}(R_l)}(\ww) \; := \\
\Big\{
\tuple{R_1, \dots, R_k, \pi_{sch(R_l)-B}(\hat{R}_l), \hat{p}}
\mid
\tuple{R_1, \dots, R_k, p} \in \ww,
\\
\mbox{$\hat{R}_l$ is a maximal repair of fd $\vec{A} \rightarrow sch(R_l)$},
\\
\hat{p} = p \cdot \prod_{\vec{t} \in \hat{R}_l}
  \frac{\vec{t}.B}
       {\sum_{\vec{s} \in R_l: \vec{s}.\vec{A}=\vec{t}.\vec{A}} \vec{s}.B}
\Big\}
\end{multline*}

Such a repair operation, apart from its usefulness for the purpose implicit
in its name, is a powerful way of constructing probabilistic databases from
complete relations.


\begin{example}\em
\label{ex:biased2}
Consider the example of tossing a biased coin twice. We start with a certain database
\begin{center}
\begin{tabular}{c|c@{~~}c@{~~}c}
\hline
R & Toss & Face & FProb \\
\hline
 & 1 & H & .4 \\
 & 1 & T & .6 \\
 & 2 & H & .4 \\
 & 2 & T & .6 \\
\end{tabular}
$p = 1$
\end{center}
that represents the possible outcomes of tossing the coin twice.
We turn this into a probabilistic database that represents this information using alternative possible worlds for the four outcomes using the query
$
S := \mbox{repair-key}_{\mathrm{Toss}@\mathrm{FProb}}(R).
$
The resulting possible worlds are
\begin{center}
\begin{tabular}{c|c@{~~}c}
\hline
$S^1$ & Toss & Face \\
\hline
 & 1 & H \\
 & 2 & H \\
\end{tabular}
%
\begin{tabular}{c|c@{~~}c}
\hline
$S^2$ & Toss & Face \\
\hline
 & 1 & H \\
 & 2 & T \\
\end{tabular}
\\[.7ex]
%
\begin{tabular}{c|c@{~~}c}
\hline
$S^3$ & Toss & Face \\
\hline
 & 1 & T \\
 & 2 & H \\
\end{tabular}
%
\begin{tabular}{c|c@{~~}c}
\hline
$S^4$ & Toss & Face \\
\hline
 & 1 & T \\
 & 2 & T \\
\end{tabular}
\end{center}
with probabilities
$p^{[1]} = p \cdot \frac{.4}{.4+.6} \cdot \frac{.4}{.4+.6} =
.16$, $p^{[2]}=p^{[3]}=.24$, and $p^{[4]}=.36$.
\punto
\end{example}
\end{itemize}

The fragment of probabilistic WSA which excludes the difference operation is called
{\em positive}\/ probabilistic WSA.


Computing possible and certain tuples is redundant with conf:
\begin{eqnarray*}
\mbox{poss}(R) &:=&
\pi_{sch(R)}(\mbox{conf}(R))
\\
\mbox{cert}(R) &:=& \pi_{sch(R)}(\sigma_{P=1}(\mbox{conf}(R)))
\end{eqnarray*}

%We will now use our algebra to compute tables of conditional probabilities.


\begin{example}\em
\label{ex:twotosses}
A bag of coins contains two fair coins and one double-headed coin. We take one coin out of the bag but do not look at its two faces to determine its type (fair or double-headed) for certain. Instead we toss the coin twice to collect evidence about its type.


\begin{figure}
\begin{center}
\begin{tabular}{@{~}c|c@{~}c@{~}}
\hline
Coins & Type & Count \\
\hline
 & fair         & 2 \\
 & 2headed & 1 \\
\\
\end{tabular}
\hspace{2mm}
\begin{tabular}{@{~}c|ccc@{~}}
\hline
Faces & Type & Face & FProb \\
\hline
 & fair    & H & .5 \\
 & fair    & T & .5 \\
 & 2headed & H &  1 \\
\end{tabular}
\hspace{2mm}
\begin{tabular}{@{~}c|c@{~}}
\hline
Tosses & Toss \\
\hline
 & 1 \\
 & 2 \\
\\
\end{tabular}

\medskip

\begin{tabular}{c|c}
\hline
$R^f$ & Type \\
\hline
 & fair          \\
\end{tabular}
\hspace{2mm}
\begin{tabular}{c|c}
\hline
$R^{dh}$ & Type \\
\hline
 & 2headed \\
\end{tabular}
%
\nop{
Corresponding U-RDB:
\begin{tabular}{c|cc|c}
\hline
$U_R$ & V & D & Type \\
\hline
 & c & fair    & fair      \\
 & c & 2headed & 2headed \\
\end{tabular}
%
\hspace{2mm}
%
\begin{tabular}{c|ccc}
\hline
$W$ & V & D & P \\
\hline
 & c & fair    & $2/3$ \\
 & c & 2headed & $1/3$ \\
\end{tabular}
} % end nop

\medskip

\nop{
\begin{tabular}{c@{~}|@{~}c@{~~}c@{~~}c@{~~}c}
\hline
$S^f$ & Face & FProb & Toss \\
\hline
 & H & .5 & 1 \\
 & T & .5 & 1 \\
 & H & .5 & 2 \\
 & T & .5 & 2 \\
\end{tabular}
\hspace{2mm}
\begin{tabular}{c@{~}|@{~}c@{~~}c@{~~}c@{~~}c}
\hline
$S^{dh}$ & Face & FProb & Toss \\
\hline
 & H &  1 & 1 \\
 & H &  1 & 2 \\
\end{tabular}

\begin{tabular}{@{~}c@{~}|@{~}c@{~~}c@{~}|@{~}c@{~~}c@{~~}c@{~~}c@{~}}
\hline
$U_S$ & V & D & Face & FProb & Toss \\
\hline
 & c & fair    & H & .5 & 1 \\
 & c & fair    & T & .5 & 1 \\
 & c & fair    & H & .5 & 2 \\
 & c & fair    & T & .5 & 2 \\
 & c & 2headed & H &  1 & 1 \\
 & c & 2headed & H &  1 & 2 \\
\end{tabular}
\begin{tabular}{@{~}c@{~}|@{~}c@{~~}c@{~~}c@{~}}
\hline
$W$ & V & D & P \\
\hline
 & c & fair    & $2/3$ \\
 & c & 2headed & $1/3$ \\
\end{tabular}
} % end nop




\nop{
\begin{tabular}{@{~}c@{~}|@{~}c@{~~}c@{~~}c@{~~}c@{~}|@{~}c@{~~}c@{~}}
\hline
$U_T$ & $V_1$ & $D_1$ & $V_2$ & $D_2$ & Toss & Face \\
\hline
 & c & fair      & f.1  & H & 1 & H \\
 & c & fair      & f.1  & T & 1 & T \\
 & c & fair      & f.2  & H & 2 & H \\
 & c & fair      & f.2  & T & 2 & T \\
 & c & 2headed   & 2h.1 & H & 1 & H \\
 & c & 2headed   & 2h.2 & H & 2 & H \\
\end{tabular}
\hspace{0.5mm}
\begin{tabular}{@{~}c@{~}|@{~}c@{~~}c@{~~}c@{~}}
\hline
$W$ & V & D & P \\
\hline
 & c    & fair      & $2/3$ \\
 & c    & 2headed   & $1/3$ \\
 & f.1  & H         & .5 \\
 & f.1  & T         & .5 \\
 & f.2  & H         & .5 \\
 & f.2  & T         & .5 \\
 & 2h.1 & H         & 1 \\
 & 2h.2 & H         & 1 \\
\end{tabular}

Now there are five possible worlds.
} % end nop



\begin{tabular}{ccc}
\begin{tabular}{@{~}c@{~}|@{~}c@{~}c@{~~}c@{~}}
\hline
$S^{f.HH}$ & Type & Toss & Face \\
\hline
& fair & 1 & H \\
& fair & 2 & H \\
\end{tabular}
&
\begin{tabular}{@{~}c@{~}|@{~}c@{~}c@{~~}c@{~}}
\hline
$S^{f.HT}$ & Type & Toss & Face \\
\hline
& fair & 1 & H \\
& fair & 2 & T \\
\end{tabular}
&
\begin{tabular}{@{~}c@{~}|@{~}c@{~}c@{~~}c@{~}}
\hline
$S^{dh}$ & Type & Toss & Face \\
\hline
& 2headed & 1 & H \\
& 2headed & 2 & H \\
\end{tabular}
\\
$p^{f.HH}=1/6$ & $p^{f.HT}=1/6$ & $p^{dh}=1/3$
\\[1ex]
\begin{tabular}{@{~}c@{~}|@{~}c@{~}c@{~~}c@{~}}
\hline
$S^{f.TH}$ & Type & Toss & Face \\
\hline
& fair & 1 & T \\
& fair & 2 & H \\
\end{tabular}
&
\begin{tabular}{@{~}c@{~}|@{~}c@{~}c@{~~}c@{~}}
\hline
$S^{f.TT}$ & Type & Toss & Face \\
\hline
& fair & 1 & T \\
& fair & 2 & T \\
\end{tabular}
&
\\
$p^{f.TH}=1/6$ & $p^{f.TT}=1/6$ &
\end{tabular}

\medskip

\begin{tabular}{@{~}c@{~}|@{~}c@{~~}c@{~}}
\hline
Ev & Toss & Face \\
\hline
 & 1 & H \\
 & 2 & H \\
\end{tabular}
\hspace{2mm}
\begin{tabular}{c@{~}|@{~}c@{~~}c}
\hline
$Q$ & Type & P \\
\hline
 & fair    & $(1/6)/(1/2) = 1/3$ \\
 & 2headed & $(1/3)/(1/2) = 2/3$ \\
\end{tabular}
\end{center}

\caption{Tables of Example~\ref{ex:twotosses}.}
\label{fig:twotosses_tables}
\end{figure}


We start out with a complete database (i.e., a relational database, or a probabilistic database with one possible world of probability 1) consisting of
three relations, Coins, Faces, and Tosses (see Figure~\ref{fig:twotosses_tables} for all tables used in this example).
%
We first pick a coin from the bag and model that
the coin be either fair or double-headed.
In probabilistic WSA this is expressed as
\[
R := \mbox{repair-key}_{\emptyset@\mathrm{Count}}(\mathrm{Coins}).
\]

This results in a probabilistic database of two possible worlds,
\[
\{ \tuple{\mbox{Coins}, \mbox{Faces}, R^f, p^{f}=2/3},
   \tuple{\mbox{Coins}, \mbox{Faces}, R^{dh}, p^{dh}=1/3} \}.
\]

The possible outcomes of tossing the coin twice can be modeled as
\[
S := \mbox{repair-key}_{\mathrm{Toss@FProb}}(
   R \bowtie \mathrm{Faces} \times \mathrm{Tosses}).
\]
This turns the two possible worlds into five, since there are four possible outcomes of tossing the fair coin twice, and only one for the double-headed coin.

Let $T := \pi_{\mathrm{Toss}, \mathrm{Face}}(S)$.
The posterior probability
that a coin of type $x$ was picked, given the {\em evidence}\/ $Ev$ (see Figure~\ref{fig:twotosses_tables}) that both tosses result in
H, is
\[
\Pr[x \in R \mid T = Ev] = \frac{\Pr[x \in R \land T = Ev]}{\Pr[T=Ev]}.
\]
Let $A$ be a relational algebra expression for the Boolean query $T=Ev$.
Then we can compute a table of pairs
$\tuple{x, \Pr[x \in R \mid T = Ev]}$ as
\[
Q := \pi_{\mathrm{Type}, P_1/P_2 \rightarrow P}(\rho_{P \rightarrow P_1}(\mbox{conf}(R \times A)) \times \rho_{P \rightarrow P_2}(\mbox{conf}(A))).
\]

The prior probability that the chosen coin was fair was 2/3; after taking the
evidence from two coin tosses into account, the posterior probability
$\Pr$[the coin is fair $|$ both tosses result in H] is only 1/3.
Given the evidence from the coin tosses, the coin is now more likely to be double-headed.
\punto
\end{example}


\begin{example}\em
\label{ex:2}
%
We redefine the query of Example~\ref{ex:twotosses} such that
repair-key is only applied to certain relations.
Starting from the database obtained by computing $R$, with its two possible worlds, we perform the query
$
S_0 := \mbox{repair-key}_{\mathrm{Type,Toss@FProb}}
      (\mathrm{Faces} \times \mathrm{Tosses})
$
to model the possible outcomes of tossing the chosen coin twice.
%
The probabilistic database representing these repairs consists of
eight possible worlds, with the two possible $R$ relations of Example~\ref{ex:twotosses} and, independently, four possible $S_0$ relations.
%
\nop{
$
\tuple{\mbox{Coins}, \mbox{Faces}, R^{ct \in \{f, dh\}}, S_0^j, p^{ct}/4},
$
with the four possible relations $S^j_0$
%
\begin{center}
\begin{tabular}{cc}
\begin{tabular}{@{~}c@{~}|@{~}c@{~~}c@{~~}c@{~~}c@{~}}
\hline
$S_0^1$ & Type & Toss & Face \\
\hline
& fair     &  1   & H \\
& fair     &  2   & H \\
& 2headed  &  1   & H \\
& 2headed  &  2   & H \\
\end{tabular}
&
\begin{tabular}{@{~}c@{~}|@{~}c@{~~}c@{~~}c@{~~}c@{~}}
\hline
$S_0^2$ & Type & Toss & Face \\
\hline
& fair     &  1   & T \\
& fair     &  2   & H \\
& 2headed  &  1   & H \\
& 2headed  &  2   & H \\
\end{tabular}
\\[2ex]
\\
\begin{tabular}{@{~}c@{~}|@{~}c@{~~}c@{~~}c@{~~}c@{~}}
\hline
$S_0^3$ & Type & Toss & Face \\
\hline
& fair     &  1   & H \\
& fair     &  2   & T \\
& 2headed  &  1   & H \\
& 2headed  &  2   & H \\
\end{tabular}
&
\begin{tabular}{@{~}c@{~}|@{~}c@{~~}c@{~~}c@{~~}c@{~}}
\hline
$S_0^4$ & Type & Toss & Face \\
\hline
& fair     &  1   & T \\
& fair     &  2   & T \\
& 2headed  &  1   & H \\
& 2headed  &  2   & H \\
\end{tabular}
\end{tabular}
\end{center}
%
For instance, the world with relations $R^{f}$ and $S_0^1$ has probability $2/3 \cdot 1/4 = 1/6$.
} % end nop
%
Let $S := R \bowtie S_0$.
While we now have eight possible worlds rather than five, the four worlds in which the double-headed coin was picked
all agree on $S$ with the one world in which the double-headed coin was picked in Example~\ref{ex:twotosses}, and the sum of their probabilities is the same as the probability of that world.
It follows that the new definition of $S$ is equivalent to the one of
Example~\ref{ex:twotosses} and the rest of the query is the same.
%
% and the new definition of $S$ can be used interchangeably with
% the one from Example~\ref{ex:twotosses}.
\punto
\end{example}




\paragraph{Discussion}
%
The repair-key operation admits an interesting class of queries:
Like in Example~\ref{ex:twotosses}, we can start with a probabilistic database of prior probabilities, add further evidence (in Example~\ref{ex:twotosses}, the result of
the coin tosses) and then compute interesting posterior probabilities. The adding of further
evidence may require extending the hypothesis space first. For this, the repair-key operation is
essential. Even though our goal is not to update the database, we have to be able to introduce uncertainty just to be able to model new evidence -- say, experimental data.
Many natural and important probabilistic database queries cannot be expressed without the repair-key operation. The coin tossing example was admittedly a toy example (though hopefully easy to understand). Real applications such as diagnosis or processing scientific data involve technically similar questions.

Regarding our desiderata, it is quite straightforward to see that probabilistic WSA is generic (3):
see also the proof for the non-probabilistic language in \cite{AKO07ISQL}. It is clearly a data transformation query language (4) that supports powerful queries for defining databases. The repair-key operation is our construct for uncertainty introduction (5). The evaluation efficiency (1) of probabilistic WSA is studied in Section~\ref{sect:theory}. Expressiveness (2) is best demonstrated by the ability of a language to satisfy many relevant use cases. While there are no agreed upon expressiveness benchmarks for probabilistic databases yet, this manual provides numerous examples that are closely related to natural use cases.






\section{Representing Probabilistic Data}
\label{sect:representation}


This section discusses the method used for representing and storing probabilistic
data and correlations in MayBMS.
We start by motivating the problem of finding a practical representation system.


\begin{example}\em
\label{ex:census}
Consider a census scenario, in which a large number of individuals manually fill in forms.
The data in these forms subsequently has to be put into a database, but no matter whether this is done automatically using OCR or by hand, some uncertainty may remain about the correct values for some of the answers. Below are two simple filled in forms. Each one contains the
social security number, name, and marital status of one person.

\begin{center}
\epsfig{width=6cm,file=census.eps}
\end{center}

The first person, Smith, seems to have checked marital status ``single'' after first mistakenly checking ``married'', but it could also be the opposite.
%
%, if the user wanted to point out the correct answer by filling the box.
%
The second person, Brown, did not answer the marital status question.
The social security numbers also have several possible readings. Smith's could be 185 or 785 (depending on whether Smith originally is from the US or from Europe) and Brown's may either be 185 or 186.

In an SQL database, uncertainty can be managed using null values, using a table
\begin{center}
\begin{tabular}{c|ccc}
\hline
(TID) & SSN & N & M \\
\hline
$t_1$ & null & Smith & null \\
$t_2$ & null & Brown & null \\
\end{tabular}
\end{center}

Using nulls, information is lost about the values considered possible for the various fields. Moreover, it is not possible to express correlations such as that, while social security numbers may be uncertain, no two distinct individuals can have the same. In this example, we can exclude the case that both Smith and Brown have social security number 185.
Finally, we cannot store probabilities for the various alternative possible worlds.
\punto
\end{example}


This leads to three natural desiderata for a representation system:
(*) Expressiveness, that is, the power to represent all (relevant) probabilistic databases,
(*) succinctness, that is, space-efficient storage of the uncertain data,
and (*) efficient real-world query processing.

Often there are many rather (but not quite) independent
local alternatives in probabilistic data, which multiply up to a very large number of possible worlds.
For example, the US census consists of many dozens of questions for about 300 million individuals.
Suppose forms are digitized using OCR and the resulting data contains just
two possible readings for 0.1\% of the answers before cleaning.
Then, there are on the order of $2^{10,000,000}$ possible worlds, and each one will take
close to one Terabyte of data to store.
Clearly, we need a way of representing this data that is much better than a naive enumeration of possible worlds.

Also, the repair-key operator of probabilistic world-set algebra in general causes an exponential increase in the number of possible worlds.

There is a trade-off between succinctness on one hand and efficient processing on the other.
Computing confidence conf$(Q)$ of conjunctive queries $Q$ on tuple-independent databases
is \#P-hard -- one such hard query \cite{dalvi07efficient} (in datalog notation \cite{AHV95}) is
\[
Q \leftarrow R(x), S(x,y), T(y).
\]
At the same time, much more expressive queries can be evaluated efficiently on nonsuccinct representations (enumerations of possible worlds) \cite{AKO07ISQL}. Query evaluation in probabilistic databases is not hard because of the presence of probabilities, but because of the succinct storage of alternative possible worlds!
We can still have the goal of doing well in practice.


\paragraph{Conditional tables}
\index{C-tables}
MayBMS uses a purely relational representation system for probabilistic databases called {\em U-relational databases}\/, which is based
on probabilistic versions of the classical {\em conditional tables}\/ (c-tables) of the database literature \cite{IL1984}.
Conditional tables are a relational representation system based on the notion of {\em labeled null values}\/ or {\em variables}\/, that is, null values that have a name. The name makes it possible to use the same variable $x$ in several fields of a database, indicating that the value of $x$ is unknown but must be the same in all those fields in which $x$ occurs. Tables with variables are also known as {\em v-tables}\/.


Formally, c-tables are v-tables extended by a column for holding a local condition. That is, each tuple of a c-table has a Boolean condition constructed using ``and'', ``or'', and ``not'' from atomic conditions of the form $x = c$ or $x = y$, where $c$ are constants and $x$ and $y$ are {\em variables}\/.
Possible worlds are determined by functions $\theta$ that map each variable that occurs in at least one of the tuples or local conditions in the c-tables of the database to a constant. The database in that possible world is obtained by (1) selecting those tuples whose local condition $\phi$ satisfies the variable assignment $\theta$, i.e., that becomes true if each variable $x$ in $\phi$ is replaced by $\theta(x)$, (2) replacing all variables $y$ in the value fields of these tuples by $\theta(y)$, and (3) projecting away the local condition column. For example, the following c-table represents the possible
worlds for the census forms:

\begin{center}
	\begin{tabular}{c|c|c|c|c}
		\hline
		R & SSN & N & M & cond\\\hline
		  & 185 & Smith & y & $x = 1$\\
		  & 785 & Smith & y & $x = 2$\\
		  & 785 & Smith & y & $x = 3$\\
		  & 186 & Brown & z & $x = 1$\\
		  & 185 & Brown & z & $x = 2$\\
		  & 186 & Brown & z & $x = 3$\\
	\end{tabular}
\end{center}

The variables $y$ and $z$ have domains $\{1,2\}$ and $\{1,2,3,4\}$, respectively and encode the marital statuses of the two persons, and variable
$x$ with domain $\{1,2,3\}$ is used to encode the uniqueness of the social security constraint. Indeed, under any valuation $\theta$ the tuples having social security status of 185 do not have their local conditions satisfied at the same time.

Conditional tables are sometimes defined to include a notion of {\em global condition}\/, which
we do not use: We want each probabilistic database to have at least one possible world. We can encode the same information as above using the following c-table with global condition $\Phi = (u\neq v)$, where $u: dom(u) = \{185,785\}, v: dom(v) = \{185,186\}$ are the variables holding the social security numbers:

\begin{center}
	\begin{tabular}{c|c|c|c|c}
		\hline
		R & SSN & N & M & cond\\\hline
		  & u & Smith & y & true \\
		  & v & Brown & z & true \\
	\end{tabular}
\end{center}


Conditional tables are a so-called {\em strong representation system}\/: They are closed under the application of relational algebra queries. The set of worlds obtained by evaluating a relational algebra query in each possible world represented by a conditional table can again be straightforwardly represented by a conditional table. Moreover, the local conditions are in a sense the most natural and simple formalism possible to represent the result of queries on data with labeled nulls.
%The local conditions just represent the information necessary to preserve correctness and can also be understood to be just data provenance information \cite{BSHW2006}.

%For nonprobabilistic uncertain databases, c-tables are the {\em gold standard}\/ that any
%formalism for representing uncertain data {\em and}\/ supporting relational algebra
%query processing should be measured against.


\paragraph{U-Relational Databases}
%
In our model, probabilistic databases are finite sets of possible worlds with probability weights. It follows that each variable naturally has a finite domain, the set of values it can take across all possible worlds. This has several consequences.
%
First, variables can be considered {\em finite random variables}\/.
Second, only allowing for variables to occur in local conditions,
but not in attribute fields of the tuples,
means no restriction of expressiveness.
Moreover, we may assume without loss of generality that each atomic condition is of the form $x=c$ (i.e., we never have to compare variables).

If we start with a c-table in which each local condition is a conjunction of no more than $k$ atomic conditions, then a positive relational algebra query on this uncertain database will result in a c-table in which each local condition is a conjunction of no more than $k'$ atoms, where $k'$ only depends on $k$ and the query, but not on the data. If $k$ is small, it is reasonable to actually hard-wire it in the schema, and represent local conditions by $k$ pairs of columns to store atoms of the form $x=c$.

These are the main ideas of our representation system, U-relations.
Random variables are assumed independent in the {\em current}\/ MayBMS system, but as we will see, this means no restriction of generality.
Nevertheless, it is one goal of future work to support
graphical models for representing more correlated joint probability distributions below our U-relations. This would allow us to represent {\em learned}\/ distributions in the form of e.g.\ Bayesian networks directly in the system (without the need to map them to local conditions) and run queries on top, representing the inferred correlations using local conditions. The latter seem to be better suited for representing the incremental correlations constructed by queries.

One further idea employed in U-relational databases is to use vertical partitioning for representing {\em attribute-level uncertainty}\/, i.e., to allow to decompose tuples in case several fields of a tuple are independently uncertain.


\begin{example}\em
\label{ex:urelation}
The following set of tables is a U-relational database representation for the census data scenario
of Example~\ref{ex:census}, extended by suitable probabilities for the various alternative values the fields can take (represented by table $W$).
%
\begin{center}
\begin{tabular}{c}
\begin{tabular}{l|cc|c|c}
\hline
$U_{R[SSN]}$ & V & D & TID & SSN \\
\hline
& $x$ & $1$ & $t_1$ & 185 \\
& $x$ & $2$ & $t_1$ & 785 \\
& $y$ & $1$ & $t_2$ & 185 \\
& $y$ & $2$ & $t_2$ & 186 \\
\end{tabular}
\\[15mm]
\begin{tabular}{l|cc|c|c}
\hline
$U_{R[M]}$ & V & D & TID & M \\
\hline
& $v$ & $1$ & $t_1$ & 1 \\
& $v$ & $2$ & $t_1$ & 2 \\
& $w$ & $1$ & $t_2$ & 1 \\
& $w$ & $2$ & $t_2$ & 2 \\
& $w$ & $3$ & $t_2$ & 3 \\
& $w$ & $4$ & $t_2$ & 4 \\
\end{tabular}
\end{tabular}
%
\hspace{5mm}
%
\begin{tabular}{c}
\begin{tabular}{l|c|c}
\hline
$U_{R[N]}$ & TID & N \\
\hline
      & $t_1$ & Smith \\
      & $t_2$ & Brown \\
\end{tabular}
\\[10mm]
\begin{tabular}{l|cc|l}
\hline
$W$ & V & D & P \\
\hline
& $x$ & $1$ & .4 \\
& $x$ & $2$ & .6 \\[.8ex]
& $y$ & $1$ & .7 \\
& $y$ & $2$ & .3 \\[.8ex]
& $v$ & $1$ & .8 \\
& $v$ & $2$ & .2 \\[.8ex]
& $w$ & $1$ & .25 \\
& $w$ & $2$ & .25 \\
& $w$ & $3$ & .25 \\
& $w$ & $4$ & .25 \\
\end{tabular}
\end{tabular}
\end{center}
%\punto
\end{example}


\index{U-relation}
Formally, a U-relational database consists of a
set of independent random variables with finite domains (here, $x,y,v,w$),
a set of U-relations, and a ternary table $W$ (the {\em world-table}\/) for representing distributions. The $W$ table stores, for each variable, which values it can take and with what probability.
%
The schema of each U-relation consists of
a {\em set}\/ of pairs ($V_i, D_i$) of {\em condition columns}\/ representing variable assignments and a set of {\em value columns}\/ for representing the data values of tuples.

The semantics of U-relational databases is as follows.
Each possible world is identified by a valuation $\theta$ that assigns one
of the possible values to each variable.
The probability of the possible world is the product of weights of the
values of the variables.
A tuple of a U-relation, stripped of its condition columns, is in a given possible
world if its variable assignments are consistent with $\theta$.
Attribute-level uncertainty is achieved through \cby{vertical decompositioning}, so one of the value columns is used for storing tuple ids and undoing the vertical decomposition on demand.


\begin{example}\em
Consider the U-relational database of Example~\ref{ex:urelation} and
the possible world
\[
\theta = \{ x \mapsto 1, y \mapsto 2, v \mapsto 1, w \mapsto 1 \}.
\]
The probability weight of this world is $.4 \cdot .3 \cdot .8 \cdot .25 = .024$. By removing all the tuples
whose condition columns are inconsistent with $\theta$ and projecting away the condition columns,
we obtain the relations
\begin{center}
\begin{tabular}{l|l|l}
\hline
$R[SSN]$ & TID & SSN \\
\hline
& $t_1$ & 185 \\
& $t_2$ & 186 \\
\end{tabular}
%
\hspace{3mm}
%
\begin{tabular}{l|l|l}
\hline
$R[M]$ & TID & M \\
\hline
& $t_1$ & 1 \\
& $t_2$ & 1 \\
\end{tabular}
%
\hspace{3mm}
%
\begin{tabular}{l|l|l}
\hline
$R[N]$ & TID & N \\
\hline
      & $t_1$ & Smith \\
      & $t_2$ & Brown \\
\end{tabular}
\end{center}
which are just a vertically decomposed version of $R$ in the chosen possible world.
That is, $R$ is $R[SSN] \bowtie R[M] \bowtie R[N]$ in that possible world.
\punto
\end{example}


\paragraph{Properties of U-relations}
U-relational databases are a {\em complete}\/ representation system for (finite) probabilistic databases \cite{AJKO2008}. This means that any probabilistic database can be represented in this formalism. In particular, it follows that U-relations are closed under query evaluation using any generic query language, i.e., starting from a represented database, the query result can again be represented as a U-relational database.
Completeness also implies that any (finite) correlation structure among tuples can be represented, despite the fact that we currently assume that the random variables that our correlations are constructed from (using tuple conditions) are independent: The intuition that some form of graphical model for finite distributions may be more powerful (i.e., able to represent distributions that cannot be represented by U-relations) is {\em false}\/.


\nop{
\paragraph{Historical Note}
The first prototype of MayBMS \cite{AKO07WSD, AKO07WSDb, OKA2008} did not use U-relations for representations, but a different representation system called {\em world-set decompositions}\/
\cite{AKO07WSD}. These representations are based on factorizations of the space of possible worlds. They can also be thought of as shallow Bayesian networks.
The problem with this approach is that some selection operations can cause an exponential blowup of the representations. This problem is not shared by U-relations, even though they are strictly more succinct than world-set decompositions.
This was the reason for introducing U-relations in \cite{AJKO2008} and developing a new prototype of MayBMS based on U-relations.
} % end nop




\section{Conceptual Evaluation and Rewritings}
\label{sect:theory}


This section gives a complete solution for efficiently evaluating a large fragment of probabilistic world-set algebra using relational database technology. Then we discuss the evaluation of the remaining operations of probabilistic WSA, namely difference and tuple confidence. Finally, an overview of known worst-case computational complexity results is given.


\paragraph{Translating queries down to the representation relations}
Let $\textit{rep}$ be the {\em representation function}\/, which
maps a U-relational data\-base to the set of possible worlds it represents.
Our goal is to give a reduction that maps
any positive relational algebra query $Q$ over probabilistic databases represented as U-relational
databases \textit{T} to an equivalent positive relational algebra query
$\overline{Q}$ of polynomial size such that
\[
\textit{rep}(\overline{Q}(T)) =
   \{Q({\cal A}^i) \mid {\cal A}^i \in \textit{rep}(T)\}
\]
where the ${\cal A}^i$ are relational database instances (possible worlds)
or, as a commutative diagram,
\vspace*{1em}
  \begin{center}
 % \begin{psmatrix}[colsep=8em,rowsep=5em,nodesepA=3pt,nodesepB=3pt]
 %   $T$                                & $\overline{Q}(T)$\\
 %   $\{{\cal A}^1,\ldots,{\cal A}^n\}$ &
 %   $\{Q({\cal A}^1),\ldots,Q({\cal A}^n)\}$
 % %
 %   \ncline{->}{1,1}{2,1}<{\textit{rep}}
 %   \ncline{->}{2,1}{2,2}^{$Q$}
 %   \ncline{->}{1,1}{1,2}^{$\overline{Q}$}
 %   \ncline{->}{1,2}{2,2}>{\textit{rep}}
 % \end{psmatrix}
  \end{center}


The following is such a reduction, which maps the
operations of positive relational algebra, poss, and repair-key
to relational algebra over U-relational representations:
\begin{eqnarray*}
\Bracks{R \times S} &:=&
  \pi_{(U_R.\overline{VD} \cup U_S.\overline{VD}) \rightarrow \overline{VD}, sch(R), sch(S)}( \\
&& \quad U_R \bowtie_{U_R.\overline{VD} \,\mathrm{consistent \, with}\, U_S.\overline{VD}} U_S)
\\
\Bracks{\sigma_\phi R} &:=& \sigma_\phi(U_R)
\\
\Bracks{\pi_{\vec{B}} R} &:=& \pi_{\overline{VD}, \vec{B}}(R)
\\
\Bracks{R \cup S} &:=& U_R \cup U_S
\\
\Bracks{\mbox{poss}(R)} &:=& \pi_{sch(R)}(U_R).
\end{eqnarray*}
The consistency test for conditions can be expressed simply using Boolean conditions (see Example~\ref{ex:proc_urel}, and \cite{AJKO2008}).
Note that the product operation, applied to two U-relations of $k$ and $l$ $(V_i, D_i)$ column pairs, respectively, returns a U-relation with $k+l$ $(V_i, D_i)$ column pairs.

For simplicity, let us assume that the elements of $\pi_{\tuple{\vec{A}}}(U_R)$ are not yet used
as variable names. Moreover, let us assume that the $B$ value column of $U_R$, which is to provide weights for the alternative values of the columns $sch(R) - (\vec{A} \cup B)$ for each tuple $\vec{a}$ in
$\pi_{\tuple{\vec{A}}}(U_R)$, are probabilities, i.e., sum up to one for each $\vec{a}$ and do not first have to be normalized as described in the definition of the semantics of repair-key in Section~\ref{sect:pwsa}.
The operation $S := \mbox{repair-key}_{\vec{A}@B}(R)$ for complete relation $R$ is translated as
\[
U_S := \pi_{\tuple{\vec{A}} \rightarrow V, \tuple{(sch(R)-\vec{A}) - \{B\}} \rightarrow D, sch(R)} U_R
\]
with
\[
W := W \cup \pi_{\tuple{\vec{A}} \rightarrow V,
                 \tuple{(sch(R) - \vec{A}) - \{B\}} \rightarrow D,
                 B \rightarrow P} U_R.
\]
Here, $\tuple{\cdot}$ turns tuples of values into atomic values that can be
stored in single fields.

That is, repair-key starting from a complete relation is just a
projection/copying of columns, even though we may create an
exponential number of possible worlds.

%Of course, we have to assure that the variables we introduce are new, but this is easy to do
%by concatenating the $\vec{A}$ values with a number generated using e.g.\ a logical clock or
%database sequence object.


\begin{example}\em
Consider again the relation $R$ of Example~\ref{ex:biased2}, which represents information about tossing a biased coin twice, and the query
%
$S := \mbox{repair-key}_{\mathrm{Toss}@\mathrm{FProb}}(R)$.
The result is
%
\begin{center}
\begin{tabular}{c|cc|ccc}
\hline
$U_S$ & V & D & Toss & Face & FProb \\
\hline
 & 1 & H & 1 & H & .4 \\
 & 1 & T & 1 & T & .6 \\
 & 2 & H & 2 & H & .4 \\
 & 2 & T & 2 & T & .6 \\
\end{tabular}
\hspace{5mm}
\begin{tabular}{c|ccc}
\hline
$W$ & V & D & P \\
\hline
 & 1 & H & .4 \\
 & 1 & T & .6 \\
 & 2 & H & .4 \\
 & 2 & T & .6 \\
\end{tabular}
\end{center}
as a U-relational database.
\punto
\end{example}


The projection technique only works if the relation that repair-key is applied to is certain.
However, this means no loss of generality (cf.\ \cite{Koch2008-SO}, and see also Example~\ref{ex:2}).

The next example demonstrates the application of the rewrite rules to compile a query down to relational algebra on the U-relations.


\begin{example}\em
\label{ex:proc_urel}
We revisit our census example with U-relations $U_{R[SSN]}$ and $U_{R[N]}$.
We ask for possible names of persons who have SSN 185,
\[
\mbox{poss}(\pi_N(\sigma_{SSN=185}(R))).
\]
To undo the vertical partitioning, the query
is evaluated as
\[
\mbox{poss}(\pi_N(\sigma_{SSN=185}(R[SSN] \bowtie R[N]))).
\]
We rewrite the query using our rewrite rules into
\[
\pi_N(\sigma_{SSN=185}(U_{R[SSN]}
   \bowtie_{\psi\wedge\phi} U_{R[N]})),
\]
where
$\psi$ ensures that we only generate tuples that occur in some worlds,
\[
\psi := (U_{R[SSN]}.V = U_{R[N]}.V \Rightarrow
    U_{R[SSN]}.D=U_{R[N]}.D),
\]
and $\phi$ ensures that the vertical partitioning is correctly undone,
\[
\phi := (U_{R[SSN]}.TID = U_{R[N]}.TID).
\]

\vspace{-5mm}

\punto
\end{example}



\paragraph{Properties of the relational-algebra reduction}
The relational algebra rewriting down to positive relational algebra on U-relations has
a number of nice properties. First, since relational algebra has PTIME (even AC$_0$) data complexity, the query language of positive relational algebra, repair-key, and poss on probabilistic databases represented by U-relations has the same.
The rewriting is in fact a {\em parsimonious translation}\/: The number of algebra operations does not increase and each of the operations selection, projection, join, and union remains of the same kind. Query plans are hardly more complicated than
the input queries. As a consequence, we were able to observe that off-the-shelf relational database query optimizers
do well in practice \cite{AJKO2008}.



\section{Asymptotic Efficiency}


We have seen in the previous section that for all but two operations of
probabilistic world-set algebra, there is a very efficient
solution that builds on relational database technology.
These remaining operations are confidence computation and relational
algebra difference.


\paragraph{Approximate confidence computation}
\index{Confidence approximation}
%
To compute the confidence in a tuple of data values occurring possibly in several tuples
of a U-relation, we have to compute the probability of the disjunction of the local conditions of all these tuples. We have to eliminate duplicate tuples because we are interested in the  probability of the data tuples rather than some abstract notion of tuple identity that is really an artifact of our representation. That is, we have to compute the probability of a
DNF, i.e., the sum of the weights of the worlds identified with valuations $\theta$ of the random variables such that the DNF becomes true under $\theta$. This problem is \#P-complete
\cite{GGH1998,dalvi07efficient}. The result is not the sum of the probabilities of the individual conjunctive local conditions, because they may, intuitively, ``overlap''.

\begin{example}\em
Consider
a U-relation with schema $\{V,D\}$ (representing a nullary relation) and two tuples
$\tuple{x,1}$, and $\tuple{y,1}$, with the $W$ relation from Example~\ref{ex:urelation}.
Then the confidence in the nullary tuple $\tuple{}$ is $\Pr[x \mapsto 1 \lor y \mapsto 1] =
\Pr[x \mapsto 1] + \Pr[y \mapsto 1] - \Pr[x \mapsto 1 \land y \mapsto 1] = .82$.
\punto
\end{example}


Confidence computation can be efficiently approximated by Monte Carlo simulation \cite{GGH1998,dalvi07efficient,Koch2008}.
The technique is based on the Karp-Luby fully poly\-no\-mi\-al-time randomized approximation scheme (FPRAS) for counting the number of solutions to a DNF formula \cite{KL1983,KLM1989,DKLR2000}.
%
There is an efficiently computable unbiased estimator that in expectation returns the probability $p$ of a DNF of $n$ clauses (i.e., the local condition tuples of a Boolean U-relation) such that computing the average of a polynomial number of such Monte Carlo steps (= calls to the Karp-Luby unbiased estimator) is an $(\epsilon, \delta)$-approximation for the probability: If the average $\hat{p}$ is taken over at least $\lceil 3 \cdot n \cdot \log(2/\delta)/\epsilon^2 \rceil$ Monte Carlo steps, then $\Pr\big[ |p - \hat{p}| \ge \epsilon \cdot p \big] \le \delta$. The paper \cite{DKLR2000} improves upon this  by determining smaller numbers (within a constant factor from optimal) of necessary iterations to achieve an $(\epsilon, \delta)$-approximation.



%The assert operation is an update operation - a knowledge compilation operation --
%that modifies the database; In queries, assert can be replaced by computing conditional
%confidences, which can again be efficiently approximated.



\paragraph{Avoiding the difference operation}
%
Difference $R-S$ is conceptually simple on c-tables.
Without loss of generality, assume that $S$ does not contain tuples
$\tuple{\vec{a}, \psi_1}, \dots, \tuple{\vec{a}, \psi_n}$ that are duplicates if the local conditions are disregarded. (Otherwise, we replace them
by $\tuple{\vec{a}, \psi_1 \lor \dots \lor \psi_n}$.)
For each tuple $\tuple{\vec{a}, \phi}$ of $R$, if
$\tuple{\vec{a}, \psi}$ is in $S$ then output
$\tuple{\vec{a}, \phi \land \neg \psi}$; otherwise, output
$\tuple{\vec{a}, \phi}$.
Testing whether a tuple is possible in the result of a query involving difference is already NP-hard \cite{AKG1991}. For U-relations, we in addition have to turn $\phi \land \neg \psi$
into a DNF to represent the result as a U-relation.
This may lead to an exponentially large output and a very large number of $\vec{V}\vec{D}$ columns may be required to represent the conditions.
For these reasons, MayBMS currently does not implement the difference operation.

In many practical applications, the difference operation can be avoided.
Difference is only hard on uncertain relations. On such relations, it can only lead to displayable query results in queries that close the possible worlds semantics using conf, computing a single certain relation.
%
Probably the most important application of the difference operation is for encoding universal constraints, for example in data cleaning.
But if the confidence operation is applied on top of a universal query, there is a trick that will often allow to rewrite the query into an
existential one  (which can be expressed in positive relational algebra plus conf,
without difference) \cite{Koch2008}.


\begin{example}\em
\label{ex:trick}
%
% BUG-FIX BEGIN
The example uses the census scenario and the uncertain relation $R[SSN]$
with columns TID and SSS discussed earlier; below we will call this relation
just simply $R$.
% BUG-FIX END
Consider the query of finding, for each TID $t_i$ and SSN $s$, the confidence in the statement that $s$ is the correct SSN for the individual associated with the tuple identified by $t_i$, assuming
that social security numbers uniquely identify individuals, that is, assuming that the functional dependency
$SSN \rightarrow TID$ (subsequently called $\psi$) holds.
In other words, the query asks, for each TID $t_i$ and SSN $s$, to find the probability $\Pr[\phi \mid \psi]$, where
$
\phi(t_i,s) = \exists t \in R\; t.TID=t_i \land t.SSN=s.
$
%
Constraint $\psi$ can be thought of as a data cleaning constraint that ensures that the SSN fields in no two distinct census forms (belonging to two different individuals) are interpreted as the same number.

We compute the desired conditional probabilities, for each possible pair of a TID and an SSN, as
$
\Pr[\phi \mid \psi] = \Pr[\phi \land \psi] / \Pr[\psi].
$
Here $\phi$ is existential (expressible in positive relational algebra) and $\psi$ is an equality-generating dependency (i.e., a special universal query) \cite{AHV95}.
%
The trick is to turn relational difference into the subtraction of probabilities,
$\Pr[\phi \land \psi] = \Pr[\phi] - \Pr[\phi \land \neg \psi]$ and
$\Pr[\psi] = 1 - \Pr[\neg \psi]$, where
$
\neg \psi = \exists t,t' \in R \; t.SSN = t'.SSN \land t.TID \neq t'.TID
$
is existential (with inequalities). Thus $\neg \psi$ and $\phi \land \neg \psi$ are
expressible in positive relational algebra. This works for a considerable superset of the equality-generating dependencies \cite{Koch2008}, which in turn subsume useful data cleaning constraints.

Let $R_{\neg \psi}$ be the relational algebra expression for $\neg \psi$,
\[
\pi_\emptyset(R
   \bowtie_{TID=TID' \land SSN \neq SSN'}
   \rho_{TID \rightarrow TID'; SSN \rightarrow SSN'}(R)),
\]
and let $S$ be
% BUG-FIX BEGIN
\begin{multline*}
\rho_{P \rightarrow P_\phi}(\mbox{conf}(R)) \bowtie
\rho_{P \rightarrow P_{\phi \land \neg \psi}}
\big(
\mbox{conf}(R \times R_{\neg \psi})
\; \cup \\
\pi_{TID, SSN, 0 \rightarrow P}
(\mbox{conf}(R) - \mbox{conf}(R \times R_{\neg \psi}))
\big)
\times
\rho_{P \rightarrow P_{\neg \psi}}(\mbox{conf}(R_{\neg \psi})).
\end{multline*}
% BUG-FIX END
The overall example query can be expressed as
\[
T :=
\pi_{TID, SSN, (P_\phi - P_{\phi \land \neg \psi})/(1 - P_{\neg \psi}) \rightarrow P}(S).
\]
For the example table $R$ given above, $S$ and $T$ are
\begin{center}
\begin{tabular}{l|lllll}
\hline
$S$ & TID & SSN & $P_\phi$ & $P_{\phi \land \neg \psi}$ & $P_{\neg \psi}$ \\
\hline
& $t_1$ & 185 & .4 & .28 & .28 \\
& $t_1$ & 785 & .6 & 0   & .28 \\
& $t_2$ & 185 & .7 & .28 & .28 \\
& $t_2$ & 186 & .3 & 0   & .28 \\
\end{tabular}
\hspace{5mm}
\begin{tabular}{l|lll}
\hline
$T$ & TID & SSN & P \\
\hline
& $t_1$ & 185 & 1/6 \\
& $t_1$ & 785 & 5/6 \\
& $t_2$ & 185 & 7/12 \\
& $t_2$ & 186 & 5/12 \\
\end{tabular}
\end{center}
\end{example}



\paragraph{Complexity Overview}
\index{Complexity}
%
Figure~\ref{tab:complexity} gives an overview over the known complexity results for the various fragments of probabilistic WSA.


\begin{figure}
\begin{center}
\begin{tabular}{|l|l|}
\hline
Language Fragment & Complexity \\
\hline
\hline
Pos.RA + repair-key + possible &
{\bf in AC0}
\\[.3ex]
RA + possible & {\bf co-NP-hard}
\\[.3ex]
Conjunctive queries + conf & {\bf \#P-hard}
\\[.3ex]
Probabilistic WSA & {\bf in ${\mathbf P^{\#P}}$}
\\[.5ex]
Pos.RA + repair-key + possible &
\\
+ approx.conf + egds &  {\bf in PTIME}
\\
\hline
\end{tabular}
\end{center}
%
\caption{Complexity results for (probabilistic) world-set algebra
\cite{KochBook2008}.
RA denotes relational algebra.}
\label{tab:complexity}
\end{figure}


Difference \cite{AKG1991} and confidence computation \cite{dalvi07efficient} independently make queries NP-hard.
Full probabilistic world-set algebra is essentially not harder than the language of \cite{dalvi07efficient}, even though it is substantially more expressive.

It is worth noting that repair-key by itself, despite the blowup of possible worlds, does not make queries hard. For the language consisting of positive relational algebra, repair-key, and poss, we have shown by construction that it has PTIME complexity: We have given a positive relational algebra rewriting to queries on the representations earlier in this section. Thus queries are even in the highly parallelizable complexity class AC$_0$.

The final result in Figure~\ref{tab:complexity} concerns the language consisting of the positive relational algebra operations, repair-key, $(\epsilon, \delta)$-approximation of confidence computation, and the generalized equality generating dependencies of \cite{Koch2008} for which we can rewrite difference of uncertain relations to difference of confidence values (see Example~\ref{ex:trick}). The result is that queries of that language that close the possible worlds semantics -- i.e., that use conf to compute a certain relation -- are in PTIME overall.


\chapter{The MayBMS Query and Update Language}


\section{Language Overview}
\label{sect:ql}


This section describes the query and update language of MayBMS, which is based on SQL.
In fact, our language is a generalization of SQL on classical relational databases.
To simplify the presentation, a fragment of the full language supported in MayBMS is presented here.

The representation system used in MayBMS, U-relations, has as a special case classical relational tables, that is, tables with no condition columns. We will call these tables {\em typed-certain (t-certain) tables}\/ in this section. Tables that are not t-certain are called uncertain.
Note that this notion of certainty is
purely syntactic, and
\[
\mbox{cert}(R) = \pi_{sch(R)}(\sigma_{P=1}(\mbox{conf}(R)))
\]
may well be equal to the projection of a U-relation $U_R$ to its attribute (non-condition) columns despite $R$ not being t-certain according to this definition.

\paragraph{Aggregates}
In MayBMS, full SQL is supported on t-certain tables.
Beyond t-certain tables, some restrictions are in place to assure that query evaluation is feasible. In particular, we do not support the standard SQL aggregates such as {\tt sum} or {\tt count} on uncertain relations. This can be easily justified: In general, these aggregates will produce exponentially many different numerical results in the various possible worlds, and there is no way of representing these results efficiently.  However, MayBMS supports a different set of aggregate operations on uncertain relations. These include the computations of {\em expected}\/ sums and counts (using aggregates {\tt esum} and {\tt ecount}).

Moreover, the confidence computation operation is an aggregate in the MayBMS query language.
This is a deviation from the language flavor of our algebra, but there is a justification for
this. The algebra presented earlier assumed a set-based semantics for relations, where operations
such as projections automatically remove duplicates. In the MayBMS query language, just like in SQL, duplicates have to be eliminated explicitly, and confidence is naturally an aggregate that computes a single confidence value for each group of tuples that agree on (a subset of) the non-condition columns.
By using aggregation syntax for {\tt conf} and not supporting {\tt select distinct} on uncertain relations, we avoid a need for conditions beyond the special conjunctions that can be stored with each tuple in
U-relations.

All supported aggregates on uncertain tables produce t-certain tables.

\paragraph{Duplicate tuples}
SQL databases in general support multiset tables, i.e., tables in which there may be duplicate tuples.  There is no conceptual difficulty at all in supporting multiset U-relations.  In fact, since U-relations are just relations in which some columns are interpreted to have a special meaning (conditions), just storing them in a standard relational database management system which supports duplicates in tables yields support for multiset U-relations.

\paragraph{Syntax}
The MayBMS query language is compositional and built from uncertain and t-certain queries. 
The uncertain queries are those that produce a possibly uncertain relation (represented by a U-relation with more than zero $V$ and $D$ columns). Uncertain queries can be constructed, inductively, from t-certain queries, {\tt select-from-where} queries over uncertain tables, the multiset union of uncertain queries (using the SQL {\tt union} construct), and the {\tt repair-key} and {\tt pick-tuples} statements that can be specified as follows
\begin{verbatim}
   repair key <attributes> in 
   (<t-certain-query> | <t-certain-relation>)
   [weight by <expression>];
   
   pick tuples from
   <t-certain-query> | <t-certain-relation>
   [independently] 
   [with probability <expression>];
\end{verbatim}
Note that {\tt repair-key} is a query, rather than an update statement.
Details on these constructs can be found in Section~\ref{sec:langref}, Language reference.

The {\tt select-from-where} queries may use any t-certain subqueries in the conditions, plus uncertain subqueries in atomic conditions of the form
\begin{verbatim}
   <tuple> in <uncertain-query>
\end{verbatim}
that occur positively in the condition. (That is, if the condition is turned into DNF, these literals are not negated.)

\nop{
Uncertain queries also support a construct {\tt tconf()} that can be used in the select list and outputs, for each tuple selected, its confidence. Applied to a multiset U-relation, this operation does not eliminate duplicates and compute aggregate confidence values as {\tt conf} does, but outputs the probability of the (conjunctive) condition for each of the tuples in the multiset.
} % end nop

The t-certain queries (i.e., queries that produce a t-certain table) are given by 
\begin{itemize}
\item
all constructs of SQL on t-certain tables and t-certain subqueries, extended by a new aggregate
\begin{verbatim}
   argmax(<argument-attribute>, <value-attribute>)
\end{verbatim}
which outputs one of the {\tt argument-attribute} values in the current group (determined by the group-by clause) whose tuples have a maximum {\tt value-attribute} value within the group. Thus, this is the typical argmax construct from mathematics added as an SQL extension.

\item
{\tt select-from-where-group-by} on uncertain queries using the {\tt possible} construct for computing possible tuples, or the aggregates {\tt conf}, {\tt esum}, and {\tt ecount}, but none of the standard SQL aggregates. There is an exact and an approximate version of the {\tt conf} aggregate. The
latter takes two parameters $\epsilon$ and $\delta$ (see the earlier discussion of the Karp-Luby FPRAS).
\end{itemize}


The aggregates {\tt esum} and {\tt ecount} compute expected sums and counts across groups of tuples.
While it may seem that these aggregates are at least as hard as confidence computation (which is \#P-hard), this is in fact not so.
These aggregates can be efficiently computed exploiting linearity of expectation.
A query
\begin{verbatim}
   select A, esum(B) from R group by A;
\end{verbatim}
is equivalent to a query
\begin{verbatim}
   select A, sum(B * P) from R' group by A;
\end{verbatim}
where {\tt R'} is obtained from the U-relation of {\tt R} by 
replacing each local condition $V_1, D_1$, $\dots$, $V_k$, $D_k$ by the probability
$\Pr[V_1=D_1 \land \dots \land V_k=D_k]$, not eliminating duplicates.
That is, expected sums can be computed efficiently tuple by tuple, and only require to determine the probability of a conjunction, which is easy, rather than a DNF of variable assignments
as in the case of the {\tt conf} aggregate.
The {\tt ecount} aggregate is a special case of {\tt esum} applied to a column of ones.


\begin{example}\em
\label{ex:coins_sql}
The query of Example~\ref{ex:twotosses} can be expressed in the query language of MayBMS as follows.
Let {\tt R} be {\tt repair key in Coins weight by Count} and let {\tt S} be
\begin{verbatim}
select R.Type, Toss, Face
from (repair key Type, Toss in (select * from Faces, Tosses)
      weight by FProb) S0, R
where R.Type = S0.Type;
\end{verbatim}

It is not hard to verify that $\pi_{\mathrm{Toss}, \mathrm{Face}}(S) \neq Ev$
exactly if there exist tuples $\vec{s} \in S, \vec{t} \in Ev$ such that
$\vec{s}.\mathrm{Toss}=\vec{t}.\mathrm{Toss}$ and
$\vec{s}.\mathrm{Face} \neq \vec{t}.\mathrm{Face}$.
Let {\tt C} be
\begin{verbatim}
select S.Type from S, Ev
where S.Toss = Ev.Toss and S.Face <> Ev.Face;
\end{verbatim}

Then we can compute {\tt Q} using the trick of Example~\ref{ex:trick} as
% BUG-FIX BEGIN
\begin{verbatim}
select Type, (P1-P2)/(1-P3) as P
from (select Type, conf() as P1 from S group by Type) Q1,
     ((select Type, conf() as P2 from C group by Type)
      union
      (
         (select Type, 0 as P2 from Coins)
         except
         (select Type, 0 as P2 from
             (select Type, conf() from C group by Type) Dummy)
      )) Q2,
     (select conf() as P3 from C) Q3
where Q1.Type = Q2.Type;
\end{verbatim}
% BUG-FIX END

The argmax aggregate can be used to compute maximum-a-posteriori (MAP) and maximum-likelihood estimates.
For example,
the MAP coin type
\[
\mbox{argmax}_{\mathrm{Type}} \; \Pr[\mbox{evidence is twice heads} \land \mbox{coin type is Type}]
\]
can be computed as
{\tt select argmax(Type, P) from Q}
because the normalizing factor {\tt (1-P3)} has no impact on argmax. Thus, the answer in this example
is the double-headed coin. (See table $Q$ of Figure~\ref{fig:twotosses_tables}: The fair coin has $P=1/3$, while the double-headed coin has $P=2/3$.)

The maximum likelihood estimate
\[
\mbox{argmax}_{\mathrm{Type}} \; \Pr[\mbox{evidence is twice heads} \mid \mbox{coin type is Type}]
\]
can be computed as
\begin{verbatim}
select argmax(Q.Type, Q.P/R'.P) 
from Q, (select Type, conf() as P from R) R'
where Q.Type = R'.Type;
\end{verbatim}
Here, again, the result is 2headed, but this time with likelihood
1. (The fair coin has likelihood 1/4).
%
\punto
\end{example}


\paragraph{Supported Queries}
\index{Supported Queries}
MayBMS supports full SQL on t-certain tables. In addition it supports a large subset of SQL on t-uncertain tables, with even more features supported when fragments of the uncertain query involve t-certain subqueries. The following restrictions apply:
\begin{itemize}
    \item
    Exact aggregates and duplicate elimination using {\tt distinct} in a select statement are supported as long as the from clause subqueries and the subqueries in the where condition are t-certain.
    \item
    If a t-certain subquery Q in the where condition of a select statement contains references to t-uncertain tables, then the containing query is supported if Q is not correlated with it.
    \item
    The set operations {\tt except} and {\tt union} with duplicate elimination are supported when both the left and the right argument are t-certain queries.
    \item
    {\tt repair-key} and {\tt pick-tuples} are supported on t-certain queries.
\end{itemize}

Restrictions on the update statements are discussed below.

\paragraph{Updates}
\index{Updates}
%
MayBMS supports the usual schema modification and update statements of SQL. In fact, our use of U-relations makes this quite easy.
An insertion of the form
\begin{verbatim}
   insert into <uncertain-table> (<uncertain-query>);
\end{verbatim}
is just the standard SQL insertion for tables we interpret as U-relations. Thus, the table inserted into must have the right number (that is, a sufficient number) of condition columns.
Schema-modifying operations such as
\begin{verbatim}
   create table <uncertain-table> as (<uncertain-query>);
\end{verbatim}
are similarly straightforward.
A deletion
\begin{verbatim}
   delete from <uncertain-table>
   where <condition>;
\end{verbatim}
admits conditions that refer to the attributes of the current tuple and may use t-certain subqueries.
One can also update an uncertain table with an update statement 
\begin{verbatim}
   update <uncertain-table>
   set <attribute> = <expr> [,...]
   where <condition>;
\end{verbatim}
where the set list does not modify the condition columns and the where condition satisfies the same conditions as that of the delete statement. MayBMS allows users to insert a constant tuple by specifying values for the data columns in an insert statement:
\begin{verbatim}
   insert into <uncertain-table> [<attribute_list>] <tuple>;
\end{verbatim}


\section{Language Reference}
\label{sec:langref} 

We next discuss the extensions to SQL by MayBMS.
For a description of the standard SQL constructs please see the Postgres SQL language reference available at

\url{http://www.postgresql.org/docs/8.3/interactive/sql-commands.html}


\subsection{repair-key}
\textbf{Syntax:}
\begin{verbatim}
   repair key <attributes> in 
   (<t-certain-query> | <t-certain-relation>)
   [ weight by <expression> ]
\end{verbatim}

\noindent \textbf{Description:}
The {\tt repair-key} operation turns a {\em t-certain-query}\/
(or, as a special case, a {\em t-certain-relation}\/) into the set of worlds consisting of all possible
{\em maximal repairs}\/ of key $attributes$. A repair of key $\vec{A}$ in 
relation $R$ is a subset of $R$ for which $\vec{A}$ is a key.
We say that relation $R'$ is a {\em maximal repair}\/ of a functional dependency 
for relation $R$ if $R'$ is a maximal subset of $R$ which satisfies that 
functional dependency. The numerically-valued $expression$ is used for 
weighting the newly created alternative repairs.
If the {\tt weight by} clause is omitted, a uniform probability distribution is assumed among all tuples with 
the same key. Suppose there are $n$ tuples sharing the same key, each of them is 
associated with a probability of $1/n$. If the weight is specified by $expression$, 
the value of $expression$ will be the probability of the tuple before normalization. 
Suppose there are $n$ tuples sharing the same key, tuple $t_i$ is associated 
with probability $expression_i$ / $\sum_{k=1}^n expression_k$. In either case, 
the sum of the probabilities among all tuples with the same key is 1. 
There will be an error message if the value of $expression$ in any tuple is 
negative. The tuples for which probability is 0 are ignored and not included in any resulting possible world.

{\tt repair-key} can be placed wherever a select statement is allowed in SQL. 
See Section~\ref{sect:pwsa} for more details on {\tt repair-key}.

\noindent \textbf{Example:}
Suppose $Customer$ is a certain
relation with columns $ID$ and $name$, the following query performs a {\tt repair-key} operation on column $ID$ in $Customer$: 

\begin{verbatim}
   repair key ID in Customer;
\end{verbatim}

Suppose $Accounts$ is a certain relation with columns $ID$ and $account$, the following is an example of {\tt repair-key} operation on column $ID$ in the output of a certain query: 

\begin{verbatim}
   repair key ID in 
   (select * from Customer natural join Accounts);
\end{verbatim}



\subsection{pick-tuples}
\textbf{Syntax:}
\begin{verbatim}
    pick tuples from 
    <t-certain-query> | <t-certain-relation>
    [independently] 
    [with probability <expression>];
\end{verbatim}

\noindent \textbf{Description:}
%
The {\tt pick-tuples} operation generates the set of worlds which can be obtained from a {\it t-certain-query} or a {\it t-certain-relation} by selecting a subset of the tuples of that query or relation. In the current version of MayBMS, the presence of {\tt independently} does not affect query evaluation. It is the default; in the future, MayBMS may be extended by other options.

By default, every tuple in a possible world is associated with probability 0.5. If {\tt with probability} $expression$ is specified, the numerical value of $expression$ is the probability of the tuple. Note that only values in (0,1] are valid. There will be an error message if the value of $expression$ is negative or larger than 1. Tuples for which $expression$ are 0 are ignored. 

{\tt pick-tuples} can be placed wherever a select statement is allowed in SQL. 

\subsection{possible}
\noindent \textbf{Syntax:}
\begin{verbatim}
    select possible <attributes> from <query> | <relation>;
\end{verbatim}

\noindent \textbf{Description:}
The operation {\tt possible} selects the set of tuples appearing in at least one possible world. This construct is a shortcut for the query which selects all distinct tuples with confidence greater than zero:
\begin{verbatim}
    select distinct <attributes> from
    (select <attributes>, tconf() as conf from <query> | <relation>
     where conf > 0) Q;
\end{verbatim}


\noindent \textbf{Example:}
Suppose R and S are uncertain relations, the following query displays distinct pairs (A,B) with positive probabilities.  
\begin{verbatim}
	select possible A, B from R, S; 
\end{verbatim}


\subsection{Confidence computation and approximate aggregates}

{\tt argmax}, {\tt conf}, {\tt aconf}, {\tt tconf}, {\tt esum} and {\tt ecount} are functions introduced by MayBMS. Following is the summary of the functions. \\

\begin{small}
\begin{tabular}{|l|l|}
\hline
Name & Brief Description  \\
\hline
argmax(argument, value) & Returns the argument with the maximum value.	 \\ \hline
conf() & Returns the exact confidence of distinct tuples.	 \\ \hline
conf(approach, $\epsilon$) & Returns the approximate confidence of distinct tuples.	 \\ \hline
aconf($\epsilon$, $\delta$) & Returns the approximate confidence of distinct tuples.	 \\ \hline
tconf() & Returns the exact confidence of tuples.	\\ \hline
esum(attribute) & Returns the expected sum over distinct tuples.	 \\ \hline
ecount(attribute) & Returns the expected count over distinct tuples.	 \\ \hline
\end{tabular}
\end{small}

\setcounter{secnumdepth}{3}

\subsubsection{argmax(argument-attribute, value-attribute)}

Outputs an {\tt argument-attribute} value in the current group (determined by the group-by clause) whose tuples have a maximum {\tt value-attribute} value within the group. If there are several tuples sharing the same maximum {\tt value-attribute} value with different {\tt argument-attribute} values, an arbitrary value among them is returned. For example, 
\begin{verbatim}
select location, argmax(date, temperature)
from weather_reports
group by location; 
\end{verbatim}
retrieves one of the dates with the highest temperature for each location.

{\tt argmax} can be used on all relations and queries.

\subsubsection{conf()}

\noindent \textbf{Syntax:}
\begin{verbatim}
	select <attribute | conf()> [, ...]
	from <query> | <relation>
	group by <attributes>; 
\end{verbatim}

\noindent \textbf{Description:}
Computes for each possible {\em distinct}\/ tuple of attribute values of the target list that occurs in an uncertain relation in at least one possible world, the sum of the probabilities of the worlds in which it occurs. {\tt conf} can only be used on a t-uncertain query or a t-uncertain relation and the output of the query is a t-certain relation.

\noindent \textbf{Example:}
Suppose weather\_forecast is an uncertain relation storing information regarding weather prediction, the following query computes the probability of each weather condition for each location:
\begin{verbatim}
	select location, weather, conf()
	from weather_forecast
	group by location, weather; 
\end{verbatim}

\subsubsection{tconf()}
\noindent \textbf{Syntax:}
\begin{verbatim}
	select <attribute | tconf()> [, ...]
	from <query> | <relation>;
\end{verbatim}

\noindent \textbf{Description:}
Computes for each possible tuple the sum of the probabilities of the worlds where it appears. ${\tt tconf()}$ is different from ${\tt conf()}$ in that it does not eliminate duplicates. {\tt tconf} can only be used on a t-uncertain query or a t-uncertain relation and the output of the query is a t-certain relation.

\subsubsection{conf(approach, $\epsilon$)}

\noindent \textbf{Syntax:}
\begin{verbatim}
	select <attribute | conf(<approach>, <epsilon>)> [, ...]
	from <query> | <relation>	
	group by <attributes>; 
\end{verbatim}

\noindent \textbf{Description:}
Computes for each possible {\em distinct}\/ tuple of the target list that occurs in at least one possible world, the $approximate$ sum of the probabilities of the worlds in which it occurs. {\tt approach} specifies the approximation approach, namely, `R' and `A' are relative and absolute approximation, respectively. Let $p$ be the exact sum (computed by {\tt conf()}) and $\hat{p}$ be the approximate sum (computed by {\tt conf(approach, $\epsilon$)}), the approximation has the following property:
\begin{itemize}
\item Relative approximation: $|p - \hat{p}| \le \epsilon \cdot p$
\item Absolute approximation: $|p - \hat{p}| \le \epsilon$
\end{itemize}

 {\tt conf(approach, $\epsilon$)} can only be used on a t-uncertain query or a t-uncertain relation and the output of the query is a t-certain relation.

\subsubsection{aconf($\epsilon$, $\delta$)}

\noindent \textbf{Syntax:}
\begin{verbatim}
	select <attribute | aconf(<epsilon>, <delta>)> [, ...]
	from <query> | <relation>	
	group by <attributes>; 
\end{verbatim}

\noindent \textbf{Description:}
Computes for each possible {\em distinct}\/ tuple of the target list that occurs in at least one possible world, the $approximate$ sum of the probabilities of the worlds in which it occurs. Let $p$ be the exact sum (computed by {\tt conf}) and $\hat{p}$ be the approximate sum (computed by {\tt aconf}), the approximation has the following property: $\Pr\big[ |p - \hat{p}| \ge \epsilon \cdot p \big] \le \delta$.

See the earlier discussion of the Karp-Luby FPRAS for more details. {\tt aconf} can only be used on a t-uncertain query or a t-uncertain relation and the output of the query is a t-certain relation. \\

\noindent \textbf{Remark:}
Although both {\tt conf(approach, $\epsilon$)} and {\tt aconf} output approximate confidence for distinct tuples, there are three major differences between them:
\begin{itemize}
\item The underlying techniques for {\tt conf(approach, $\epsilon$)} and 
{\tt aconf} are d-tree approximation algorithm~\cite{OHK2010} and Karp-Luby FPRAS, respectively. 
The former is a deterministic algorithm and outputs the same probability if the 
databases and queries are identical while the latter is randomized and likely to output
different probabilities even if the databases and queries are identical. 
\item {\tt conf(approach, $\epsilon$)} provides both absolute and relative approximation while {\tt aconf} only allows the latter.
\item {\tt conf(approach, $\epsilon$)} outputs an $\epsilon$-approximation certainly while {\tt aconf} guarantees it with probability $1-\delta$. 
\end{itemize}

\subsubsection{esum and ecount}

\noindent \textbf{Syntax:}
\begin{verbatim}
	select <attribute | esum(<attribute>) | ecount()> [, ...]
	from <query> | <relation>
	group by <attributes>; 
\end{verbatim}

\noindent \textbf{Description:}
{\tt esum} and {\tt ecount} compute expected sum and count, respectively. {\tt ecount} can take zero or one argument, and the number of arguments does not affect the results. {\tt esum} and {\tt ecount} can only be used on a t-uncertain query or a t-uncertain relation and the output of the query is a t-certain relation.

\noindent \textbf{Example:}
The following query  computes the expected total rainfall of seven days for each location:
\begin{verbatim}
	select location, esum(rainfall)
	from rainfall_forecast
	where date >= '2010-10-01' and date <= '2010-10-07'
	group by location; 
\end{verbatim}









%\chapter{Design and Implementation of the MayBMS System}
\chapter{MayBMS Internals}
\label{sect:system}


\paragraph{Representations, relational encoding, and query optimization}
%
Our representation system, U-relations, is basically implemented as described earlier, with one small exception. With each pair of columns $V_i$, $D_i$ in the condition, we also store a column $P_i$ for the probability weight of alternative $D_i$ for variable $V_i$, straight from the $W$ relation. While the operations of relational algebra, as observed earlier, do not use probability values, confidence computation does. This denormalization (the extension by $P_i$ columns) removes the need to look up any probabilities in the $W$ table in our exact confidence computation algorithms.

Our experiments show that the relational encoding of positive relational algebra which is possible for U-relations is so simple -- it is a parsimonious transformation, i.e., the number of relational algebra operations is not increased -- that the standard Postgres query optimizer actually does well at finding good query plans (see \cite{AJKO2008}).


\paragraph{Approximate confidence computation}
%
MayBMS implements both an approximation algorithm and several exact algorithms for confidence computation. The approximation algorithm is a combination of the Karp-Luby unbiased estimator for DNF counting \cite{KL1983,KLM1989} in a modified version adapted for confidence computation in probabilistic databases (cf.\ e.g.\ \cite{Koch2008}) and the Dagum-Karp-Luby-Ross optimal algorithm for Monte Carlo estimation \cite{DKLR2000}. The latter is based on sequential analysis and determines the number of invocations of the Karp-Luby estimator needed to achieve the required bound by running the estimator a small number of times to estimate its mean and variance. We actually use the probabilistic variant of a version of the Karp-Luby estimator described in the book \cite{Vazirani2001} which computes fractional estimates that have smaller variance than the zero-one estimates of the classical Karp-Luby estimator.


\paragraph{Exact confidence computation}
\index{Confidence computation, exact}
%
Our exact algorithm for confidence computation is described in \cite{KO2008}. It is based on an extended version of the Davis-Putnam procedure \cite{DP1960} that is the basis of the best exact Satisfiability solvers in AI. Given a DNF (of which each clause is a conjunctive local condition), the algorithm employs a combination of variable elimination (as in Davis-Putnam) and decomposition of the DNF into independent subsets of clauses (i.e., subsets that do not share variables), with cost-estimation heuristics for choosing whether to use the former (and for which variable) or the latter.


%\newcommand{\mystackrel}[2]{\stackrel{#1}{#2}}
%\newcommand{\mystackrel}[2]{\begin{tabular}{c} $#1$ \\ $#2$ \end{tabular}}


\begin{figure}[!]
\[
\begin{tabular}{c}
\begin{tabular}{@{~}l@{~}|@{~}c@{~~}c@{~~}c@{~~}c@{~}}
\hline
$U$ & $V_1$ & $D_1$ & $V_2$ & $D_2$ \\
\hline
& $x$ & 1 & $x$ & 1 \\ 
& $x$ & 2 & $y$ & 1 \\ 
& $x$ & 2 & $z$ & 1 \\ 
& $u$ & 1 & $v$ & 1 \\ 
& $u$ & 2 & $u$ & 2 \\ 
\end{tabular}
\\
{~}~{~}
\\
\begin{tabular}{l|ccc}
\hline
$W$ & V & D & P \\
\hline
& $x$ &  1 &   .1 \\
& $x$ &  2 &   .4 \\
& $x$ &  3 &   .5 \\
& $y$ &  1 &   .2 \\
& $y$ &  2 &   .8 \\
& $z$ &  1 &   .4 \\
& $z$ &  2 &   .6 \\
& $u$ &  1 &   .7 \\
& $u$ &  2 &   .3 \\
& $v$ &  1 &   .5 \\
& $v$ &  2 &   .5 \\
\end{tabular}
\end{tabular}%
\hspace{-2mm}%
\relax\parbox{0.8\textwidth}{\relax%
\[
\pstree[nodesep=1pt,levelsep=10ex,treesep=3.5em]{\TR{\framebox{$\stackrel{0.7578}{\otimes}$}}}
{
  \pstree{\TR{\framebox{$\stackrel{0.308}{\oplus}$}}^{\{x,y,z\}}}
  {
    \TR{\framebox{$\stackrel{1.0}{\emptyset}$}}^{x \stackrel{.1}{\mapsto} 1}
    \pstree{\TR{\framebox{$\stackrel{0.52}{\otimes}$}}_{x \stackrel{.4}{\mapsto} 2}}
    {
      \pstree{\TR{\framebox{$\stackrel{0.2}{\oplus}$}}^{\{y\}}}
      {
        \TR{\framebox{$\stackrel{1.0}{\emptyset}$}}^{y \stackrel{.2}{\mapsto} 1}
      }
      \pstree{\TR{\framebox{$\stackrel{0.4}{\oplus}$}}_{\{z\}}}
      {
        \TR{\framebox{$\stackrel{1.0}{\emptyset}$}}_{z \stackrel{.4}{\mapsto} 1}
      }
    }
  }
  \pstree{\TR{\framebox{$\stackrel{0.65}{\oplus}$}}_{\{u,v\}}}
  {
    \pstree{\TR{\framebox{$\stackrel{0.5}{\oplus}$}}^{\ u \stackrel{.7}{\mapsto} 1}}
    {
      \TR{\framebox{$\stackrel{1.0}{\emptyset}$}}_{v \stackrel{.5}{\mapsto} 1}
    }
    \TR{\framebox{$\stackrel{1.0}{\emptyset}$}}_{u \stackrel{.3}{\mapsto} 2}
  }
}
\]}
\]

\caption{Exact confidence computation.}
\label{fig:exact-conf}
\end{figure}



\begin{example}\em
Consider the U-relation $U$ representing a nullary table and
the $W$ table of Figure~\ref{fig:exact-conf}.
The local conditions of $U$ are
\[
\Phi = \{ \{x\mapsto 1\}, \{x\mapsto 2, y\mapsto 1\}, \{x\mapsto 2, z\mapsto 1\},
   \{u\mapsto 1, v\mapsto 1\}, \{u\mapsto 2\} \}.
\]

The algorithm proceeds recursively. We first choose to exploit the fact that the $\Phi$ can be split into two independent sets, the first using only the variables $\{x, y, z\}$ and the second only using $\{u,v\}$.
We recurse into the first set and eliminate the variable $x$. This requires us to consider two cases, the alternative values 1 and 2 for $x$ (alternative 3 does not have to be considered because in each of the clauses to be considered, $x$ is mapped to either 1 or 2. In the case that $x$ maps to 2, we eliminate $x$ from the set of clauses that are compatible with the variable assignment $x \mapsto 2$, i.e., the set
$\{\{y\mapsto 1\}, \{z\mapsto 1\} \}$, and
can decompose exploiting the independence of the two clauses. Once $y$ and $z$ are eliminated, respectively, the conditions have been reduced to ``true''. The alternative paths of the computation tree, shown in Figure~\ref{fig:exact-conf}, are processed analogously.

On returning from the recursion, we compute the probabilities of the subtrees in the obvious way. For two independent sets $S_1, S_2$ of clauses with probabilities $p_1$ and $p_2$, the probability of $S_1 \cup S_2$ is
\[
1 - (1-p_1)\cdot(1-p_2).
\]
For variable elimination branches,
the probability is the sum of the products of the probabilities of the subtrees and the probabilities of the variable assignments used for elimination.

It is not hard to verify that the probability of $\Phi$, i.e., the confidence in tuple $\tuple{}$,
is $0.7578$.
\punto
\end{example}


Our exact algorithm solves a \#P-hard problem and exhibits exponential running time in the worst case. However, like some other algorithms for combinatorial problems, this algorithm shows a clear easy-hard-easy pattern. Outside a narrow range of variable-to-clause count ratios, it very pronouncedly outperforms the (polynomial-time) approximation techniques \cite{KO2008}.
It is straightforward to extend this algorithm to condition a probabilistic database (i.e., to compute ``assert'') \cite{KO2008}.


\paragraph{Hierarchical queries}
\index{Hierarchical queries}
The tuple-independent databases are those probabilistic databases in
which, for each tuple, a probability can be given such that the tuple
occurs in the database with that probability and the tuples are
uncorrelated. It is known since the work of Dalvi and Suciu
\cite{dalvi07efficient} that there is a class of conjunctive queries,
the hierarchical queries $Q$, for which computing conf($Q$) exactly on
tuple-independent probabilistic databases is feasible in polynomial
time.

In fact, these queries can essentially be computed using SQL queries
that involve several nested aggregate-group-by queries. On the other
hand, it was also shown in \cite{dalvi07efficient} that for any
conjunctive query $Q$ that is not hierarchical, computing conf($Q$) is
\#P-hard with respect to data complexity. Dalvi and Suciu introduce
the notion of {\em safe plans}\/\index{Safe plans} that are at once
certificates that a query is hierarchical and query plans with
aggregation operators that can be used for evaluating the queries.

To deal with hierarchical queries, MayBMS runs SPROUT as part of its
query engine~\cite{OHK2008}. SPROUT extends the early work by Suciu in
three ways. First, the observation is used that in the case that a
query has a safe plan~\cite{dalvi07efficient}, it is not necessary to
use that safe plan for query evaluation. Instead, one can choose any
unrestricted query plan, not only restricted safe plans, for the
computation of the answer tuples; confidence computation is performed
as an aggregation which can be pushed down or pull up past joins in
relational query plans. Second, the aggregation function for
confidence computation is implemented as a special low-level operator
in the query engine. Finally, the fact is exploited that the
\#P-hardness result for any single nonhierarchical query of \cite{dalvi07efficient} 
only applies as long as the problem is that of evaluating the query on
an arbitrary probabilistic database of suitable schema. If further
information about permissible databases is available in the form of
functional dependencies that the databases must satisfy, then a larger
class of queries can be processed by our approach~\cite{OHK2008}.


\paragraph{Updates, concurrency control and recovery}
%
As a consequence of our choice of a purely relational representation
system, these issues cause surprisingly little difficulty.
U-relations are just relational tables and updates are just
modifications of these tables that can be expressed using the standard
SQL update operations. While the structure of the rewritings could allow for optimizations in the concurrency and recovery managers, those are currently left to the underlying DBMS.







\chapter{The MayBMS Codebase}
\label{sect:codebase}

MayBMS is currently implemented in PostgreSQL 8.3.3. Integration into
an existing full-fledged DBMS brings two major advantages. First,
integration makes it possible to reuse the internal functions and
structures of the DBMS. Secondly, it often increases the efficiency of
query processing.

Figures~\ref{fig:modified-files1} and \ref{fig:modified-files2} give a
list of source files modified or added to the original PostgreSQL
8.3.3. All modifications are explicitly marked in the source files by 
\begin{verbatim}
/* MAYBMS BEGIN */
... [some code goes here]
/* MAYBMS END */
\end{verbatim}
All files in directory \texttt{maybms} are newly created and the
others are existing files in PostgreSQL8.3.3. Header files (*.h) refer
to \texttt{src/include/directory/filename}. Source files (*.c and *.y)
refer to \texttt{src/backend/directory/filename}.

\begin{figure}[ht]
\begin{center}
\small
\begin{tabular}{|l|l|}
\hline
File & Description \\
\hline
parser/gram.y           & Adds new constructs such as repair-key and possible. \\ \hline
parser/keyword.c        & Adds necessary keywords. \\ \hline
nodes/parsenodes.h      & Adds the relation type to structure CreatStmt. \\ \hline
catalog/pg\_class.h 	& Adds an extra column specifying the type of a relation \\
catalog/pg\_attribute.h & in the catalog. \\ \hline 
nodes/copyfuncs.c		& Copying the relation type. \\ \hline
catalog/heap.c          & Execution of creating urelations.  \\ \hline
catalog/heap.h			& An argument tabletype is added to function   \\ 
catalog/toasting.c		& heap\_create\_with\_catalog in heap.h.  \\ 
commands/tablecmds.c	& All files accessing this function are modified.\\
commands/cluster.c		& \\
bootstrap/bootparse.y	& \\
executor/execMain.c		& \\ \hline
\end{tabular}
\end{center}

\vspace*{-1em}
\caption{ Files related to U-relation creation.}
\label{fig:modified-files1}
\end{figure}


\begin{figure}[ht]
\begin{center}
\small
\begin{tabular}{|l|l|}
\hline
File & Description \\
\hline
catalog/pg\_proc.h              & Registers conf, tconf, aconf, argmax, esum, ecount  \\
								& and the related functions. \\ \hline 
catalog/pg\_aggregate.h 		& Specifies the relationships between conf, aconf and  \\
                           		& the related state, final functions. \\ \hline
nodes/execnodes.h 				& Adds confidence computation states to structure AggState. \\ 
executor/nodeAgg.c      		&   \\ \hline
tcop/postgres.c        			& Access point to query rewriting. \\ \hline
maybms/conf\_comp.h           	& Prototypes for conf, tconf, aconf and their related functions.  \\ \hline 
maybms/SPROUT.c                     & Confidence computation of conf for hierarchical \\
                                & queries on tuple-independent U-relations using SPROUT. \\ \hline
maybms/tupleconf.c              & Confidence computation for tconf. \\ \hline
maybms/ws-tree.c            	& Confidence computation of conf for arbitrary \\
                            	& U-relations using ws-tree-based algorithm. \\ \hline
maybms/bitset.h                 & Auxiliary files for ws-tree-based algorithm. \\ 
maybms/bitset.c                 &  \\ \hline
maybms/aconf.c      			&    Implementation of approximate confidence computation. \\ \hline
maybms/signature.h   			& Derives signatures for hierarchical queries. \\  
maybms/signature.c      		&   \\ \hline
maybms/repair\_key.c       		& Implementation of repair-key construct by pure rewriting. \\ \hline 
maybms/pick\_tuples.c       		& Implementation of pick-tuples construct by pure rewriting. \\ \hline 
maybms/localcond.h       		& Storing the condition columns for confidence computation. \\  
maybms/localcond.c      		&    \\ \hline
maybms/argmax.c      			&    Implementation of aggregate function argmax. \\ \hline
maybms/rewrite.c      			&  Rewriting of select and create commands involving uncertainty. \\ 
maybms/rewrite\_utils.c      	&   \\ \hline
maybms/rewrite\_updates.c      	&    Rewriting of update commands (insert, delete, update). \\ \hline
maybms/supported.c     			&    Checking whether a query is supported and should be rewritten. \\ \hline
maybms/utils.h					& Utility functions. \\	 
maybms/utils.c       			&  \\	\hline                                               
\end{tabular}

\end{center}

\vspace*{-1em}
\caption{ Files related to confidence computation and query rewriting. }
\label{fig:modified-files2}
\end{figure}

\vspace*{-10em}










\chapter{Experiments}



This section reports on experiments performed with the first MayBMS release
(beta) and a benchmark consisting of two parts,
which are described in more detail in the remainder of this chapter:
%
\begin{enumerate}
\item
Computing the probability of triangles in random graphs.

\item
A modified subset of the TPC-H queries on uncertain TPC-H datasets.
\end{enumerate}


By this benchmark, we do not attempt to simulate a representative set of
use cases: the jury is still out on what such a set of use cases might be.
Instead, we focus on a benchmark that allows us to see how the performance
of MayBMS develops across releases on the two core technical problems solved
by MayBMS: polynomial-time query evaluation for the polynomial-time fragment
of our query language and the efficient approximation of query results for
queries that do not belong to the polynomial-time fragment. (Finding triangles
in random graphs is a near-canonical example of such queries.)

We will keep monitoring the development of the state of the art and will
continue to survey applications and collect use cases; we will extend or
replace this benchmark as consensus develops regarding the most important
applications of probabilistic databases.


\medskip

Experimental setup.
All the experiments reported on in this chapter were conducted on an Athlon-X2(4600+)64bit / 1.8GB / Linux2.6.20 / gcc4.1.2 machine.



\section{Random Graphs}
\subsection{Experiments with Varying Levels of Precision} 

In this experiment, we create undirected random graphs in which 
the presence of each edge is independent of that of the other edges. The probability that an edge is in the graph is 0.5 and
this applies to each edge. Then we compute the probability that there exists a triangle in the graphs using approximation.
The queries can be found in Appendix~\ref{app:randgraph}.

We report wall-clock execution times
of queries run in the PostgreSQL8.3.3 psql shell with a warm
cache obtained by running a query once and then reporting
the average execution time over three subsequent, identical executions.
Figure \ref{fig:randgraph} shows the execution time of approximation with different precision parameters for random graphs composed of 5 to 33 nodes. An ($\epsilon, \delta$) approximation has the following property: let $p$ be the exact probability and $\hat{p}$ be the approximate probability, then
$\Pr\big[ |p - \hat{p}| \ge \epsilon \cdot p \big] \le \delta$.

\begin{figure}[htp]

\begin{center}
  \begin{tabular}{ | c | c | c | c | c | c | }
    \hline
    \multirow{2}{*}{\#nodes} & \multirow{2}{*}{\#clauses} & \multicolumn{4}{|c|}{Execution Time(Seconds)} \\ \cline{3-6}
          &  & (.05,.05) & (.01,.01) & (.005,.005) &  (.001,.001)  \\ \hline
    5 & 10 & 0.01 & 0.03 & 0.11 & 2.08  \\ \hline
    6 & 20 & 0.01 & 0.08 & 0.26 & 5.27   \\ \hline
	7 & 35 & 0.02 & 0.14 & 0.46 & 9.15   \\ \hline
	8 & 56 & 0.03 & 0.22 & 0.7 & 12.49   \\ \hline
	9 & 84 & 0.04 & 0.28 & 0.85 & 14.95   \\ \hline
	10 & 120 & 0.08 & 0.44 & 1.13 & 16.19   \\ \hline
	11 & 165 & 0.15 & 0.60 & 1.60 & 17.98   \\ \hline
	12 & 220 & 0.29 & 1.24 & 2.48 & 24.31   \\ \hline
	13 & 286 & 0.55 & 2.38 & 4.74 & 35.29   \\ \hline
	14 & 364 & 0.98 & 4.26 & 8.38 & 51.51   \\ \hline
	15 & 455 & 1.56 & 6.74 & 13.29 & 73.00   \\ \hline
	16 & 560 & 2.37 & 10.26 & 19.21 & 102.97   \\ \hline
	17 & 680 & 3.46 & 14.6 & 28.76 & 144.02   \\ \hline
	18 & 816 & 4.92 & 20.49 & 41.1 & 206.18  \\ \hline
	19 & 969 & 7.03 & 28.52 & 56.43 & 291.21  \\ \hline
	20 & 1140 & 9.97 & 39.72 & 81.01 & 395.18  \\ \hline
	21 & 1330 & 14.74 & 57.13 & 123.79 &  597.86  \\ \hline
	22 & 1540 & 23.94 & 119.81 & 218.62 & 600+   \\ \hline
	23 & 1771 & 46.21 & 204.83 & 416.42 & 600+  \\ \hline
	24 & 2024 & 79.03 & 411.67 & 600+ & 600+  \\ \hline
	25 & 2300 & 115.64 & 515.65 & 600+ & 600+  \\ \hline
	26 & 2600 & 159.66 & 600+ & 600+ & 600+  \\ \hline
	27 & 2925 & 202.98 & 600+ & 600+ & 600+  \\ \hline
	28 & 3276 & 251.82 & 600+ & 600+ & 600+  \\ \hline
	29 & 3654 & 312.89 & 600+ & 600+ & 600+  \\ \hline
	30 & 4060 & 387.72 & 600+ & 600+ & 600+  \\ \hline
	31 & 4495 & 475.78 & 600+ & 600+ & 600+  \\ \hline
	32 & 4960 & 582.4 & 600+ & 600+ & 600+  \\ \hline
	33 & 5456 & 600+ & 600+ & 600+ & 600+  \\ \hline

  \end{tabular}
\end{center} 

\caption{Comparison between execution time of approximation with different precision}

\label{fig:randgraph}
\end{figure}

\subsection{Experiments with Different Edge Probabilities}  

In the previous experiments, each edge had probability 0.5. We use other values as the edge probability(all edges still have the same probability) and run the experiment again with (0.05,0.05) approximation. The SQL statements in Appendix~\ref{app:randgraph} should be modified accordingly. Let $p$ be the probability, change the following statements
\begin{verbatim}
insert into  inout values (1, 0.5); 
insert into  inout values (0, 0.5); 
\end{verbatim}
 to
\begin{verbatim}
insert into  inout values (1, p); 
insert into  inout values (0, 1 - p); 
\end{verbatim}
Figure \ref{fig:edge-prob} shows the execution time for queries of random graphs composed of 25 to 101 nodes with different fixed edge probabilities.

\begin{figure}[htp]

\begin{center}
  \begin{tabular}{ | c | c | c | c | c | }
    \hline
    \multirow{2}{*}{\#nodes} & \multirow{2}{*}{\#clauses} & \multicolumn{3}{|c|}{Execution Time(Seconds)} \\ \cline{3-5}
      	&  & p=0.5 & p=0.1 & p=0.05   \\ \hline
	25 & 2300 & 115.64 & 1.77 & 0.55   \\ \hline
	%26 & 2600 & 159.66 & 2.28 & 0.72   \\ \hline
	%27 & 2925 & 202.98 & 2.52 & 0.83   \\ \hline
	%28 & 3276 & 251.82 & 3.19 & 1.02   \\ \hline
	%29 & 3654 & 312.89 & 3.73 & 1.19   \\ \hline
	30 & 4060 & 387.72 & 4.13 & 1.35   \\ \hline
	31 & 4495 & 475.78 & 4.94 & 1.54   \\ \hline
	32 & 4960 & 582.40 & 5.72 & 1.82   \\ \hline
	33 & 5456 & 600+ & 6.87 & 2.12   \\ \hline
	%34 & 5984 & 600+ & 7.48 & 2.60  \\ \hline
	35 & 6545 & 600+ & 8.74 & 2.74  \\ \hline
	%36 & 7140 & 600+ & 9.59 & 3.12  \\ \hline
	%37 & 7770 & 600+ & 11.53 & 3.63  \\ \hline
	%38 & 8436 & 600+ & 13.38 & 3.92  \\ \hline
	%39 & 9139 & 600+ & 15.32 & 4.6  \\ \hline
	40 & 9880 & 600+ & 18.32 & 5.06  \\ \hline
	%41 & 10660 & 600+ & 20.65 & 5.76  \\ \hline
	%42 & 11480 & 600+ & 23.91 & 6.51  \\ \hline
	%43 & 12341 & 600+ & 28.44 & 7.7  \\ \hline
	%44 & 13244 & 600+ & 32.38 & 8.48  \\ \hline
	45 & 14190 & 600+ & 36.77 & 8.96  \\ \hline
	%46 & 15180 & 600+ & 41.09 & 9.99  \\ \hline
	%47 & 16215 & 600+ & 48.68 & 11.45  \\ \hline
	%48 & 17296 & 600+ & 54.66 & 12.62  \\ \hline
	%49 & 18424 & 600+ & 61.05 & 13.39  \\ \hline
	50 & 19600 & 600+ & 70.79 & 15.79  \\ \hline
	%51 & 20825 & 600+ & 80.19 & 16.09  \\ \hline
	%52 & 22100 & 600+ & 88.32 & 17.16  \\ \hline
	%53 & 23426 & 600+ & 97.99 & 19.49  \\ \hline
	%54 & 24804 & 600+ & 112.07 & 21.58  \\ \hline
	55 & 26235 & 600+ & 123.69 & 21.97  \\ \hline
	%56 & 27720 & 600+ & 138.92 & 25.73  \\ \hline
	%57 & 29260 & 600+ & 155.86 & 27.52  \\ \hline
	%58 & 30856 & 600+ & 172.39 & 29.37  \\ \hline
	%59 & 32509 & 600+ & 190.98 & 32.06  \\ \hline
	60 & 34220 & 600+ & 214.06 & 33.94  \\ \hline
	%61 & 35990 & 600+ &  & 36.97  \\ \hline
	%62 & 37820 & 600+ &  & 38.40  \\ \hline
	%63 & 39711 & 600+ &  & 42.80  \\ \hline
	%64 & 41664 & 600+ &  & 43.89  \\ \hline
	65 & 43680 & 600+ & 343.66 & 47.09  \\ \hline
	%66 & 45760 & 600+ &  & 51.56  \\ \hline
	%67 & 47905 & 600+ &  & 54.87  \\ \hline
	68 & 50116 & 600+ & 451.06 & 59.87  \\ \hline
	69 & 52934 & 600+ & 490.64 & 64.69  \\ \hline
	70 & 54740 & 600+ & 542.61 & 68.98  \\ \hline
	71 & 57155 & 600+ & 595.03 & 72.88  \\ \hline
	72 & 59640 & 600+ & 600+ & 82.30  \\ \hline
	75 & 67525 & 600+ & 600+ & 106.49  \\ \hline
	80 & 82160 & 600+ & 600+ & 154.92  \\ \hline
	85 & 98770 & 600+ & 600+ & 224.3  \\ \hline
	90 & 117480 & 600+ & 600+ & 316.28  \\ \hline
	95 & 138415 & 600+ & 600+ & 437.39  \\ \hline
	97 & 147440 & 600+ & 600+ & 510.39  \\ \hline
	98 & 152096 & 600+ & 600+ & 543.87  \\ \hline
	99 & 156849 & 600+ & 600+ & 558.44  \\ \hline
	100 & 161700 & 600+ & 600+ & 593.84  \\ \hline
	101 & 166650 & 600+ & 600+ & 600+  \\ \hline
	
  \end{tabular}
\end{center} 

\caption{Comparison between execution time of queries of random graphs with different fixed edge probabilities}

\label{fig:edge-prob}
\end{figure}

\subsection{Experiments with General Random Graphs} 

The previous experiments were conducted on undirected graphs in which every pair of nodes had a possibly present edge. However, this may not be the case in general. In many scenarios, each pair of nodes may have a certainly present, certainly absent or possibly present edge. In our following experiments, we construct such general probabilistic random graphs from data representing directed links between webpage within nd.edu domain\footnote{http://www.nd.edu/~networks/resources/www/www.dat.gz}. If a link between two pages is absent from the data, then it is also absent from our graphs. If a link is present in the data, then it is a certainly or possibly present edge in our graphs. We run again the queries computing the probabilities of existence of triangles in such graphs with (0.05,0.05) approximation. The probabilities that possibly present edges are in the graphs are randomly distributed in (0,0.1). The queries of the graph constructions and confidence computation can be found in Appendix~\ref{app:general-randgraph}. Figure \ref{fig:general-randgraph} shows the execution time for queries of such random graphs composed of 1000 to 30000 nodes.

\begin{figure}[htp]

\begin{center}
  \begin{tabular}{ | c | c | c | c | }
    \hline
    \#nodes & \#possible edges & \#clauses & Execution Time(Seconds) \\ \hline
	  	1000 & 3271 & 6367 &  4.04   \\ \hline
	  	2000 & 6446 & 12598 & 11.84    \\ \hline
	  	3000 & 9056 & 19836 &  21.88   \\ \hline
	  	4000 & 11366 & 22455 &  28.57   \\ \hline
	  	5000 & 13497 & 24574 &  31.38   \\ \hline
	  	6000 & 16095 & 25731 &  35.36   \\ \hline
	  	7000 & 17958 & 26070 &  35.82   \\ \hline
	  	8000 & 23113 & 39481 &  80.14   \\ \hline
	  	9000 & 26114 & 43369 &  115.45   \\ \hline
	  	10000 & 32975 & 51586 &  140.00   \\ \hline
	  	11000 & 35507 & 55562 &  157.34   \\ \hline
	  	12000 & 37623 & 57260 &  170.05   \\ \hline
	  	13000 & 40246 & 61060 &  197.67   \\ \hline	  	
	  	14000 & 44045 & 66530 &  225.88   \\ \hline
	  	15000 & 45434 & 66966 &  230.51   \\ \hline
	  	16000 & 47814 & 69787 &  260.70   \\ \hline
	  	17000 & 50456 & 72710 &  278.48  \\ \hline
	  	18000 & 52145 & 73043 &  280.76  \\ \hline
	  	19000 & 53849 & 73437 &  288.01  \\ \hline
	  	20000 & 55584 & 73953 &  289.30  \\ \hline	  	
	  	21000 & 57654 & 74688 &  290.37  \\ \hline
		22000 & 59274 & 74991 &  295.66   \\ \hline
		23000 & 61308 & 75954 &  296.13   \\ \hline
		24000 & 63000 & 76288 &  313.13  \\ \hline
		25000 & 65538 & 79404 &  354.95   \\ \hline
		26000 & 69741 & 89888 &  439.01   \\ \hline
		27000 & 72741 & 93016 &  479.78  \\ \hline
		28000 & 76148 & 98065 &  553.75   \\ \hline
		29000 & 79414 & 104328 &  573.24   \\ \hline
		30000 & 82714 & 107633 &  601.33   \\ \hline

  \end{tabular}
\end{center}       

\caption{Execution time of confidence computation for existence of triangles in general random graphs}

\label{fig:general-randgraph}
\end{figure}

\section{Probabilistic TPC-H}

SPROUT\footnote{http://web.comlab.ox.ac.uk/projects/SPROUT/index.html} is a part 
of the query engine of MayBMS and provides state-of-the-art techniques for efficient 
exact confidence computation. In this section, we show how TPC-H queries can 
benefit from these techniques. For each TPC-H query, we consider its largest subquery
without aggregations and inequality joins but with
conf() for specifying exact probability computation
for distinct tuples in query answers. We consider two
flavours of each of these queries: A version with original
selection attributes (again, without aggregations), and a version
where we drop keys from the selection attributes. Queries are included in the 
experiments if SPROUT's techniques can be applied to them. Our data set consists 
of tuple-independent probabilistic databases obtained from deterministic
databases produced by TPC-H 2.7.0 by associating each
tuple with a Boolean random variable and by choosing at random
a probability distribution over these variables. We perform experiments with TPC-H 
scale factor 1 (1GB database size) and evaluate the 
TPC-H-like queries mentioned above. The queries can 
be found in Appendix~\ref{app:tpch}. In addition, we compare our results 
with the reported time from \cite{OHK2008} in which SPROUT was only partially 
integrated into PostgreSQL and storing temporary relations to the disk was 
sometimes necessary. The average time shown below is obtained from ten subsequent, 
identical executions with a warm cache by running the query once.

\begin{figure}[htp]

\begin{center}
  \begin{tabular}{ | c | c | c | }
    \hline
    Query &  \multicolumn{2}{|c|}{Average Time(Seconds)} \\ \cline{2-3}
           &  Current running time & Time reported in \cite{OHK2008} \\ \hline
	1 & 8.21 & 120.13 \\ \hline
	4 & 40.57 & 39.52 \\ \hline
	12 & 17.1 & 21.94 \\ \hline
	15 & 5.5 & 3.2 \\ \hline
	B1 & 5.37 & 14.92 \\ \hline
	B4 & 31.88 & 33.02 \\ \hline
	B6 & 3.82 & 6.37 \\ \hline
	B12 & 15.91 & 18.56 \\ \hline
	B14 & 4.17 & 4.86 \\ \hline
	B15 & 4.81 & 5.24 \\ \hline
	B16 & 0.87 & 3.16 \\ \hline
	B17 & 3.25 & 2.43 \\ \hline

  \end{tabular}
\end{center} 
\caption{Current running times vs.\ running times reported in \cite{OHK2008}.
Boolean queries are prefixed by B. }

\end{figure}







\nop{
\chapter{Planned Extensions}


Planned features for future releases of MayBMS are
\begin{itemize}
\item
The relaxation of some current minor restrictions in the query language.

\item
More efficient confidence computation.

\item
A knowledge compilation operation for conditioning a probabilistic database,
i.e., removing possible worlds that do not satisfy a given constraint.

\item
Continuous probability distributions.

\item
Support for importing graphical models such as Bayesian Networks.
\end{itemize}
} % end nop


\newpage

\appendix
\chapter{Queries in Random Graph Experiments} 
\label{app:randgraph}

\begin{verbatim}
create table node (n integer);
insert into  node values (1);
insert into  node values (2);
insert into  node values (3);
insert into  node values (4);
......
insert into  node values (n-1);
insert into  node values (n); /* n is the number of nodes in the graph */

/* Here we specify the probability that an edge is in the graph. */
create table inout (bit integer, p float);
insert into  inout values (1, 0.5); /* probability that edge is in the graph */
insert into  inout values (0, 0.5); /* probability that edge is missing */

create table total_order as
(
   select n1.n as u, n2.n as v
   from node n1, node n2
   where n1.n < n2.n
);

/* This table represents all subsets of the total order over
   node as possible worlds. We use the same probability -- from
   inout -- for each edge, but in principle we could just as
   well have a different (independent) probability for each edge.
*/
create table to_subset as
(
   repair key u,v
   in (select * from total_order, inout)
   weight by p
);

create table edge0 as (select u,v from to_subset where bit=1);

select conf() as triangle_prob
from   edge0 e1, edge0 e2, edge0 e3
where  e1.v = e2.u and e2.v = e3.v and e1.u = e3.u
and    e1.u < e2.u and e2.u < e3.v;

select aconf(0.05,0.05) as triangle_prob
from   edge0 e1, edge0 e2, edge0 e3
where  e1.v = e2.u and e2.v = e3.v and e1.u = e3.u
and    e1.u < e2.u and e2.u < e3.v;

\end{verbatim}
\newpage

\chapter{Queries in General Random Graph Experiments} 
\label{app:general-randgraph}

\begin{verbatim}

drop table data0;
drop table data;

create table data0(u int, v int);
create table data(u int, v int);

/* Copy the data to a relation. */
copy data0 
from 'path_of_the_data_file/www.dat' with delimiter as ' ';

/* Since the data represents a directed graph, we need to 
   insert all tuples again with u and v swapped. 
*/
insert into data0
select v, u from data0;

/* This fetches the distinct pairs of (u,v), which represents 
   all edges of an undirected graph.  
*/
insert into data
select distinct u, v from data0;

drop table edges;
drop table edge0;

create table edges (u integer, v integer, p float4);

/* This fetches all the edges related to the nodes we intend to 
   keep in the graph. 
   '1000' in 'u < 1000  and v < 1000' is the number of nodes 
   which will appear in the graph. 
   '0.01' in 'random() < 0.01' is the proportion of certainly 
   present edges in all edges.
   '0.1' is the upper bound of the probability that a possibly 
   present edge is in the graph.
   You may change the above-mentioned three parameters in the
   experiments.
 */
insert into edges
	select u, v,
   	CASE WHEN random() < 0.01 THEN 1.0  
         ELSE random() * 0.1           
         END	      
    from data
    where u < 1000  and v < 1000 and u < v; 

/* The number of edges in the graph */
select count(*) as edge_count from edges;

/* The number of clauses in the confidence computation */	
select count(*) as clause_count from 
edges e1, edges e2, edges e3 
where  e1.v = e2.u and e2.v = e3.v and e1.u = e3.u 
		and e1.u < e2.u and e2.u < e3.v;
	
/* Creation of an uncertain relations representing the graph */	
create table edge0 as 
(pick tuples from edges independently with probability p);
	
/* Confidence computation of existence of at least 
   a triangle in the graph  
*/		
select aconf(.05,.05) as triangle_prob 
from edge0 e1, edge0 e2, edge0 e3 
where  e1.v = e2.u and e2.v = e3.v and e1.u = e3.u 
		and e1.u < e2.u and e2.u < e3.v;

\end{verbatim}
\newpage


\chapter{Probabilistic TPC-H Queries} 
\label{app:tpch}

\textbf{\large{Query 1:}}

\begin{verbatim}
select 
    l_returnflag, 
    l_linestatus, 
    conf() 
from 
    lineitem 
where 
    l_shipdate <= date '1998-09-01' 
group by 
    l_returnflag, 
    l_linestatus;
\end{verbatim}

\noindent
\textbf{\large{Query 4:}}

\begin{verbatim}
select
    o_orderpriority,
    conf()
from
    orders,
    lineitem
where
    o_orderdate >= date '1993-07-01'
    and o_orderdate < date '1993-10-01'
    and l_orderkey = o_orderkey
    and l_commitdate < l_receiptdate
group by
    o_orderpriority;
\end{verbatim}

\noindent
\textbf{\large{Query 12:}}

\begin{verbatim}
select
    l_shipmode,
    conf()
from
    orders,
    lineitem
where
    orders.o_orderkey = lineitem.l_orderkey
      and (l_shipmode = 'MAIL' 
        or l_shipmode = 'SHIP')
       and l_commitdate < l_receiptdate
       and l_shipdate < l_commitdate
       and l_receiptdate >= '1992-01-01'
       and l_receiptdate < '1999-01-01'
group by
    l_shipmode;

\end{verbatim}

\noindent
\textbf{\large{Query 15:}}

\begin{verbatim}
select
    s_suppkey,
    s_name,
    s_address,
    s_phone,
    conf()
from
    supplier,
    lineitem
where
    s_suppkey = l_suppkey
    and l_shipdate >= date '1991-10-10'
    and l_shipdate < date '1992-01-10'
group by
    s_suppkey,
    s_name,
    s_address,
    s_phone;

\end{verbatim}

\noindent
\textbf{\large{Boolean Version of Query 1:}}

\begin{verbatim}
select 
    conf()
from 
    lineitem 
where 
    l_shipdate <= date '1998-09-01';
\end{verbatim}

\noindent
\textbf{\large{Boolean Version of Query 4:}}

\begin{verbatim}
select
    conf()
from
    orders,
    lineitem
where
    o_orderdate >= date '1993-07-01'
    and o_orderdate < date '1993-10-01'
    and l_orderkey = o_orderkey
    and l_commitdate < l_receiptdate
group by
    o_orderpriority;
\end{verbatim}

\noindent
\textbf{\large{Boolean Version of Query 6:}}

\begin{verbatim}
select
    conf()
from
    lineitem
where
    l_shipdate >= '1994-01-01'
    and l_shipdate < '1995-01-01'
    and l_discount >= 0.05 
    and l_discount <= 0.07
    and l_quantity < 24;
\end{verbatim}

\noindent
\textbf{\large{Boolean Version of Query 12:}}

\begin{verbatim}
select
    conf()
from
    orders,
    lineitem
where
    orders.o_orderkey = lineitem.l_orderkey
      and (l_shipmode = 'MAIL' 
        or l_shipmode = 'SHIP')
       and l_commitdate < l_receiptdate
       and l_shipdate < l_commitdate
       and l_receiptdate >= '1992-01-01'
       and l_receiptdate < '1999-01-01'
group by
    l_shipmode;
\end{verbatim}

\noindent
\textbf{\large{Boolean Version of Query 14:}}

\begin{verbatim}
select
    conf()
from
    lineitem,
    part
where
    l_partkey = p_partkey
    and l_shipdate >= date '1995-09-01'
    and l_shipdate < date '1995-10-01';
\end{verbatim}

\noindent
\textbf{\large{Boolean Version of Query 15:}}

\begin{verbatim}
select
    conf()
from
    supplier,
    lineitem
where
    s_suppkey = l_suppkey
    and l_shipdate >= date '1991-10-10'
    and l_shipdate < date '1992-01-10';
\end{verbatim}

\noindent
\textbf{\large{Boolean Version of Query 16:}}

\begin{verbatim}
select
    conf()
from
    partsupp,
    part
where
    p_partkey = ps_partkey
    and p_brand <> 'Brand#45'
    and p_type like 'MEDIUM POLISHED%';
\end{verbatim}

\noindent
\textbf{\large{Boolean Version of Query 17:}}

\begin{verbatim}
select
    conf()
from
    lineitem,
    part
where
    p_partkey = l_partkey
    and p_brand = 'Brand#23'
     and p_container = 'MED BOX';
\end{verbatim}

\newpage



\bibliographystyle{abbrv}
\bibliography{bibtex}




\end{document}


