\chapter{Queries in General Random Graph Experiments} 
\label{app:general-randgraph}

\begin{verbatim}

drop table data0;
drop table data;

create table data0(u int, v int);
create table data(u int, v int);

/* Copy the data to a relation. */
copy data0 
from 'path_of_the_data_file/www.dat' with delimiter as ' ';

/* Since the data represents a directed graph, we need to 
   insert all tuples again with u and v swapped. 
*/
insert into data0
select v, u from data0;

/* This fetches the distinct pairs of (u,v), which represents 
   all edges of an undirected graph.  
*/
insert into data
select distinct u, v from data0;

drop table edges;
drop table edge0;

create table edges (u integer, v integer, p float4);

/* This fetches all the edges related to the nodes we intend to 
   keep in the graph. 
   '1000' in 'u < 1000  and v < 1000' is the number of nodes 
   which will appear in the graph. 
   '0.01' in 'random() < 0.01' is the proportion of certainly 
   present edges in all edges.
   '0.1' is the upper bound of the probability that a possibly 
   present edge is in the graph.
   You may change the above-mentioned three parameters in the
   experiments.
 */
insert into edges
	select u, v,
   	CASE WHEN random() < 0.01 THEN 1.0  
         ELSE random() * 0.1           
         END	      
    from data
    where u < 1000  and v < 1000 and u < v; 

/* The number of edges in the graph */
select count(*) as edge_count from edges;

/* The number of clauses in the confidence computation */	
select count(*) as clause_count from 
edges e1, edges e2, edges e3 
where  e1.v = e2.u and e2.v = e3.v and e1.u = e3.u 
		and e1.u < e2.u and e2.u < e3.v;
	
/* Creation of an uncertain relations representing the graph */	
create table edge0 as 
(pick tuples from edges independently with probability p);
	
/* Confidence computation of existence of at least 
   a triangle in the graph  
*/		
select aconf(.05,.05) as triangle_prob 
from edge0 e1, edge0 e2, edge0 e3 
where  e1.v = e2.u and e2.v = e3.v and e1.u = e3.u 
		and e1.u < e2.u and e2.u < e3.v;

\end{verbatim}
\newpage

